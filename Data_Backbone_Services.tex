\subsubsection{Scope}

The Data Backbone is a set of data storage, management and movement
resources distributed between the Base Center and NCSA. The
scope of the Data Backbone includes both files and data maintained by relational
and other database engines holding the record of the survey and used by
L1, L2, and Data Access Center services.

The Data Backbone provides read-only data service to the US and Chilean
DACs, but does not host data stores where DAC users create state. This is
done to create a hard and easily enforceable separation of technologies,
where no flaw in a DAC can corrupt the data produced by L1 and L2
production systems. For example, DAC resources such as Qserv and user databases,
colloquially know as MyDBs, are provisioned in the context of a data
access center, not the Data Backbone.

The Data Backbone ensures that designated data sets are replicated at both sites.
The data backbone provides an enclave environment that is oriented toward
protecting data by management, operational, and technical controls,
including processes such as maintaining disaster recovery copies.

\subsubsection{Overview}

\paragraph{Description}

Files in the data backbone are presented as file system mounts and data access
services. Database-resident data are presented as managed database services.

\paragraph{Objective}

\begin{itemize}

\item Replication of designated file data within LSST Data Facility sites
at NCSA and the Base Center.

\item Replication of designated relational tables and data maintained
in other database engines at NCSA and the Base Center.

\item Implementation of policy-based flows to the disaster recovery stores. At
the time of this writing, disaster recovery stores include the NCSA tape archive, CC-IN2P3,
and commercial providers.

\item Ingest of images produced by the Spectrograph, ComCam, and LSSTCam instruments.

\item Ingest of the Engineering and Facility Database and associated Large File Annex.

\item Ingest of data products from L1 and L2 production processing.

\item Ingest of data from TBD other sources, approved by a change control process.

\item Serving data to L1, L2, and other approved production processes.

\item Serving data to the US and Chilean Data Access Centers.

\item Integrity checking and other curation activates related to data integrity and
continuity of operations.

\end{itemize}

\subparagraph{Operational Concepts}

\subparagraph{Files}

Files in the Data Backbone possess path names which are subject to change
through the lifetime of the LSST project, which at the time of this writing is
seen as serving the last data release through 2034.

Robust identification of a file involves
\begin{itemize}
\item Obtaining a logical file name through querying metadata and provenance.
\item Possibly migrating a file from a medium where the file is not directly accessible,
such as tape, to a medium where the file is accessible.
\item Selecting a distinct instance of the file from possibly many replicas.
\item Accessing the file though an access method such as a file system mount or URL(s).
\end{itemize}

The project has identified several caches of data that are used in production circumstances.
The distinguishing circumstances for these caches involve quality of service requirements for
performance and availability. Absent sophisticated QoS in file systems, performance requirements
are met by controlling access to the underlying storage via caching. Availability is assured by
decoupling the cache from the database providing metadata, provenance, and location information.
Application-level cache management provides path names within the cache to the application.

Casual use of data for short periods may rely on knowledge such as file paths, but is subject
to disruption when paths are re-arranged, or should the underlying storage technology change,
such as introduction of object stores.


\subparagraph{Data Managed by Databases}

Replication of database information is specific to the database technology involved.
Databases identified as holding permanent records of the survey are in the Data Backbone
in the sense that they are instantiated in the context of a security enclave with management,
operational, and technical controls needed to assure preservation of this data, and that the
principal concern of enclave management is that data reside at the Base and at NCSA
driven by business need.

\subparagraph{Managed Lifetime of Data}
Data in the data backdone are perserved and replicated under processes
guided by an overall Data Management Policy.  Elements of the policy
include specifications for replication suporting disasater recovery,
and the permissiblity of deleteing file based data, and purgining
database records.  Examples of deletion of data likley in such a
policy are: Removal of production data from faulty production campaigns,
removal of data produced in the production system that are not part
of the retained record of the LSST suvey, and removal of data in
shakedown and debug processing.

\paragraph{Operational Scenarios}

\subparagraph{Availablity and Change Management}

Catalog-based access systems such as indicated for the Data Backbone are
limited by database availability as well as the availability of the file
store and its access methods.

Time-critical applications involving the Observatory Operations Data
Service and access to L1 templates for prompt processing protect
themselves by having caches as described above.
