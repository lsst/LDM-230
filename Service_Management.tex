
\subsubsection{Scope}
This section describes operational concepts of systems-level and service-level
management for services operated by the LSST Data Facility.
kk

\subsubsection{Overview}
This section briefly describes functions and processes of service
management that are implemented across all service and ITC layers of
the LSST Data Facility. These elements were drawn from the Information
Technology Infrastructure Library (ITIL) which is an industry-standard
vocabulary for IT service management.

IT Service Management processes include

\begin{enumerate}

\item Service Design: Building a service catalog and arranging for changes to the service offering, including internal supporting services.

\item Service Transition: Specifying needed changes, assessing the quality of proposed changes,
and controlling the order and timing of inserting changes into the system.

  \begin{itemize}

  \item \emph{Change Management} provides authorization for streams of changes to be requested, for the insertion of changes into the reliable production system, and for the assessment of the success of these changes.

  \item \emph{Release Management} interacts with a project producing a specific change to ensure that
a complete change is presented to change management for approval into the live system. Examples areas that are typically a concern are accompanying documentation and security aspects.

  \item \emph{Configuration Management} provides an accurate model of the components in the live system sufficient to understand changes, and support operations.

  \end{itemize}

\item Service Delivery: operating the current set and configuration of production services. Service delivery processes must satisfy the detailed service delivery concepts presented elsewhere in this document.

  \begin{itemize}

  \item \emph{Incident Response} is invoked when a service does not
  perform as specified according to its version.  The goal of
  incident respose is to allow work to continue, either by
  restoring the service to its original working form, or by providing
  a work-around.  Each LSST service has a level of incident
  response support, which depends on the operational criticality of the service.  For
  example, support of observatory services is likely more critical that support
  of a developer test stand.

  \item \emph{Request Response} Request reponse covers both structured
  requests which are well supported by a workflow, and unstructured
  requests.  An example of structured requests is to on-board or
  off-board a user. Examples of unstructured requests are to answer a
  question about the implementation of a service, or to produce an
  informative report.

  \item \emph{Problem Management} Not all issues with a service can be
  addressed at a given time. These issues are aggregated into a problem
  backlog. The problem back log contains unaddressed, valid issues
  know to service users, and problems generated by internal staff.
  Problem management is the process where items are selected from the
  backlog to be addressed, root cause determined, followed by a change
  request to implement a fix.

  \end{itemize}

\end{enumerate}
\subsubsection{Operational Concepts}
TBD 


