\subsubsection{Scope}
This section describes the concept of operations for hosting Data Access Center services in the US and Chile.

\subsubsection{Overview}

\paragraph{Description}

Hosting of user data access services consists of:
\begin{itemize}

\item Providing Kubernetes for the LSST Science Platform, including the SUIT portal, Jupyter Notebooks, and DAX.

\item Providing a Qserv instance for each data release.

\item Loading a Qserv with the released data to the extent not already done during data release production.

\item Providing support for users entering data into their MyDBs and entering sharded data into Qserv to the extent that is supported, as well as providing backup and disaster recovery for data entered into these databases.

\item Providing user file space and backups and disaster recovery for the file space.

\item Providing read-only access to data release products identified in the DPDD and other designated data resources, such as the Transformed EFD.

\item Providing batch computing capabilities.

\item Providing a second-tier help desk support to the primary help desk.

\item Providing for special administration considerations of the Chilean DAC outlined in the Chilean MoU.

\item Providing a security enclave appropriate for protecting LSST resources from actions of the user community.

\item Integrating the LSST Authentication and Authorization system.

\item Providing service-level monitoring to measure and assure a designated level of service.

\item Providing users with a default level of access to computing resources.

\item Providing users with a computing award with the designated amount of computing resources.

\end{itemize}

\paragraph{Objective}

The objective is to host the data access services at the required availability for authorized users.

\paragraph{Operational Context}

The operational context is the LSST LDF infrastructure at the Base and at NCSA. An important element of the context is the elasticity of the supporting computing infrastructure. While, for example, the US Data Access Center consist of about 10\% of the computing capacity at NCSA, it is desirable that the constructed system deliver to the end users 10\% of the computing capacity averaged over a year, borrowing resources from the production system when demand on the Data Access Center is high, and returning resources to the production system during times when demand is low.

It is not required that the US and Chilean Data Access Centers be mirrors of each other, as this implies capacities may need to be nearly equal, which is not a requirement of the Chilean MoU. They are mirrors of each other ffor structure and configuration.  

\paragraph{Operational Concepts}
TBD
