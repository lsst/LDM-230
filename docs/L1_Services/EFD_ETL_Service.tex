\subsubsection{Scope}

The Engineering and Facility Database (EFD) See \citeds{LTS-210} is a system used in the
context of Observatory Operations. It contains data, apart from pixel
data acquired by the Level 1 archiving systems, of interest to LSST
originating from any instrument or any operation related to
observing. The EFD is an essential record of the activities of
Observatory Operations. It contains data for which there is no
substitute, as it records raw data from supporting instruments,
instrumental conditions, and actions taking place within the
observatory.

This section describes the concept of operations for ingesting the
EFD data into the LSST Data Backbone and transforming this data
into a format suitable for offline use.

\subsubsection{Overview}

\paragraph{Description}

The Original Format EFD, maintained by Observatory Operations, is
conceived of as a collection of files and approximately 20 autonomous
relational database instances, all using the same relational database
technology. The relational tables in the Original Format EFD have a
temporal organization. This organization supports the need within
Observatory Operations to support high data ingest and access
rates. The data in the Original Format EFD is immutable, and will not
change once entered.

The EFD also includes a large file annex that holds flat files that
are part of the operational records of the survey.

\paragraph{Objective}

The prime motivation behind the EFD Transformation Service is to be able to
relate the time series data to raw images, and to hold these quantities in
a manner that is accessible using the standard DM methods for file and
relational data access.

The baseline design calls for all of the EFD relational
and flat-file material to be ingested into what is called the Transformed
EFD.

\begin{enumerate}

\item There is a need to access a substantial subset of the Original Format
EFD data in the general context of Level 2 data production and in the
Data Access Centers. This access is supported by a query-access
optimized, separately implemented relational database, generally
called the Transformed EFD. A prime consideration is to relate the
time series data to raw images.

\item To be usable in offline context, files from the Original Format EFD
need to be ingested into the LSST Data Backbone. This ingest operation
requires provenance and metadata associated with these files.

\item Because the Original format EFD is the fundamental record related to
instrumentation, the actions of observers, and related data, the data
contained within it cannot be recomputed, and in general there is no
substitute for this data. Best practice for disaster recovery is to
not merely replicate the Original Format EFD live environment, but also
to make periodic backups and ingests to a disaster recovery system.

\end{enumerate}

\paragraph{Operational Context}

Ingest of data from the Original Format EFD into the Transformed EFD
must be controlled by Observatory Operations, based on the principle
that Observatory Operations controls access to the Original Format EFD
resources. The prime framework for controlling operations is the OCS
system. Operations in this context will be controlled from the OCS
framework.

\paragraph{Risks}

The query load applied by EFD Transformation Service  on the Original Format
EFD at the Base Center may be disruptive to the primary purpose,
serving Observatory Operations.

\subsubsection{Operational Concepts}

\paragraph{Normal Operations}

\subparagraph{Original Format EFD Operations}

Observatory Operations is responsible for Original Format EFD
operations during LSST Operations. Observatory
Operations will copy database state and related operational
information into a disaster recovery store at a frequency consistent
with a Disaster Recovery Plan approved by the LSST ISO. The LSST Data
Facility will provide the disaster recovery storage resource. The DR
design procedure should consider whether normal operations may begin
prior to a complete restore of the Original Format EFD.

If future operations of the LSST telescope beyond the lifetime of the
survey do not provide for operation and access to the Original Format
EFD, the LSST Data Facility will assume custody of the Original Format
EFD and arrange appropriate service for these data (and likely move
the center of operations to NCSA) in the period of data production
following the cessation of LSST operations at the Summit and Base
Centers.

LSST staff working on behalf of any operations department will have
access to the Original Format EFD at the Base Center for one-off
studies, including studying the merits of data being loaded into the
Transformed EFD. 

\subparagraph{EFD Large File Annex Handling and Operations}

Under control of the EFD Transformation Service, the LSST Data Facility will
ingest the designated contents of the file annex of the Original
Format EFD into the data backbone. The LSST Data Facility will arrange that these
files participate in whatever is developed for disaster recovery for the
files in the Data Backbone.

These files will also participate in the general file metadata and
file-management service associated with the Data Backbone, and thus be
available using I/O methods of the LSST stack.

\subparagraph{Transformed EFD Operations}

\begin{itemize}
\item The Transformed EFD is replicated to the US DAC and the Chilean DAC.
\item LDF will extract, transform and load into the Transformed EFD pointers to files
that have been transferred from the EFD large file annex into the Data Backbone.
\item LDF will extract, transform and load all tabular data from the
Original Format EFD into the Transformed EFDs residing in the Data Backbone at
NCSA and the Base Center. Information in Transformed EFD that is private needs to be blocked on export.  
\item “Designated” data will include:
  \begin{itemize}
  \item Any quantities used in a production process.
  \item Any quantities designated by an authorized change control process.
  \end{itemize}
\item The information in the Transformed EFD is available to any authorized independent
DAC which may choose to host a copy.
\end{itemize}

\paragraph{Operational Scenarios}

\subparagraph{ETL Control}

The Extract, Transform and Load operation is under the control of Observing Operations.

\subparagraph{Disaster Recovery and DR Testing for the Original Format EFD}

Observing Operations will periodically test a restore in a disaster recovery scenario.

\subparagraph{Disaster Recovery and DR Testing for the Transformed EFD}

Since the Transformed EFD relational database is reproducible from the
Original Format EFD, disaster recovery is provided by a re-ingest from the
original format. DR testing includes re-establishing operations of the Transformed
EFD relational database and ETL capabilities from the Original Format EFD. Ingested
files from the file annex can be recovered by the general disaster recovery capabilities
of the Data Backbone.

\subparagraph{Change Control}

Upgrades to the EFD ETL Service are produced by the LSST Data Facility.
Change control of this function is coordinated with the Observatory, with the
Observatory having an absolute say about insertion and evaluation of changes.
