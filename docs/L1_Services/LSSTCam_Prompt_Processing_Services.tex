\subsubsection{Scope}
This section describes the prompt processing of raw data acquired from the main 
LSST camera by the DM system.

\subsubsection{Overview}

\paragraph{Description}
During nightly operations, the DM system acquires images from the main LSST 
camera as they are taken, and promptly processes them with codes specific to an 
observing program. 

\paragraph{Objective}
The LSSTCam Prompt Processing Services provide timely processing of newly 
acquired raw data, including QA of images, alert processing and delivery, 
returning image parameters to the Observatory, and populating the Level 1 
Database.

\paragraph{Operational Context}
Prompt Processing is a service provided by the LSST Data Facility as part of the 
Level 1 system. It is presented to Observatory Operations as an OCS-commandable 
device. The Prompt Processing Service retrieves crosstalk-corrected pixel data 
from the main LSST camera at the Base Center, builds FITS images, and sends them 
to NCSA for prompt processing.

\subsubsection{Operational Concepts}

\paragraph{Normal Operations}

\subparagraph{Science Operations}

Science data-taking occurs on nights when conditions are suitable. For
LSST, this means all clear nights, even when the full moon brightens
the night sky. Observing is directed by an automated scheduler. The
scheduler considers observing conditions, for example, the seeing, the
phase of the moon, the atmospheric transparency, and the part of the
sky near the zenith. The scheduler is also capable of receiving
external alerts, for example, announcements from LIGO of a
gravitational wave event. The scheduler also considers required
observing cadence and depth of coverage for the LSST observing
programs.

About 90\% of observing time is reserved for the LSST “wide-fast-deep”
program. In this program, observations will be on the wide-field
two-image-per-visit cadence, in which successive observations will be
in the same filter with no slew of the telescope. However, a new
program, potentially with a new filter, a larger slew, a different
observing cadence, or a different visit structure, can be scheduled at
any moment.

Another evisioned program is ``deep drilling'', where many more exposures
than the two exposure visit will be taken.

In practice, science data-taking will proceed when conditions are
suitable. Calibration data may be taken when conditions are not
suitable for further observations, with science data-taking resuming
when conditions again become suitable.

It follows that the desired behavior for science data-taking
operations is to start the Prompt Processing system at the beginning
of the night and to turn off the system after all processing for all
science observations is finished.  

The operational framework for observing discloses no future knowlege
about what exposures will be taken, until the “next visit” is
determined.

During science data-taking the Prompt Processing Service computes and
promptly returns QA parameters (referred to as “telemetry”) to the
observing system. The QA parameters are not specific to an observing
program; examples are seeing and pointing corrections derived from the
WCS. These parameters are not strictly necessary for observing to
proceed -- LSST can observe autonomously at the telescope site, if
need be. Also note that the products are quality parameters, not the
the “up-or-down” quality judgement.

The scheduler may be sensitive to a subset of these messages and may
decide on actions, but a detailed description is TBD and may vary as
the survey evolves. The scheduler can make use of these parameters even
if delivered quite late, since the scheduler uses historical data
in addition to recent h data.

The Prompt Processing system also executes code that is specific to an
observing program. For science exposures, the code is divided into
a front-end -- that is able to compute the parameters sent back  to the
observator, and a back end , Alert Production (AP), is the specific
science code that detects transient objects.

The detected transients are passed off to another service, which
records the data in a catalog which can be queried offline and sends
events to an ensemble of transient brokers. Data are transmitted to end
users either via feeds from an LSST-provisioned broker or via
community-provided alert brokers.

Alert Production runs in the context of the wide-fast-deep survey,
deep drilling program and TBD other programs. Other observing programs
may also include AP as a science code, or may have codes of their own.

\subparagraph{Calibration Operations}

In addition to collecting data for science programs, the telescope and
camera are involved in many calibration activities.

the Baseline Calibrations include flats and biases. Darks are not anticipated.
LSST has a collumanted Beam projector calibration device as well.

Nominally, a three-hour period each afternoon is scheduled for Observatory
Operations to take dark and flat calibration data. As noted above, calibration data
may be taken during the night when conditions are not suitable for
science observations. The LSST dome is specified as being
light-tight, enabling certain calibration data to be collected
whenever operationally feasible, regardless of the time of day.

Although there are standard cadences for calibration operations, the
frequency of calibration data-taking is sensitive to the stability of
the camera and telescope. Certain procedures, such as replacement of a
filter, cleaning of a mirror, and warming of the camera, may
subsequently require additional calibration operations. In general,
calibration operations will be modified over the lifetime of the
survey as understanding of the LSST telescope and camera improves.

The Prompt Processing Service computes and promptly returns QA
parameters (referred to as “telemetry”) to the observing system. Note
that the quality of calibrations needed for Prompt Processing science
operations may be less stringent than calibrations needed for other
processing, such as annual release processing.

An operations strawman, which illuminates the general need for prompt
processing, is that there are two distinct, high-level types of
calibrations.

\begin{itemize}

\item Nightly flats, biases and darks consist of about a sequence of  broad-band
flatfield exposures each camera filter, bias frames acquired from
rapid reads of an un-illuminated camera, and optional dark images
acquired from reads of an un-illuminated camera at the cadence of the
expected exposure time.  Observers will consider the collection of
these nightly calibrations as a single operational sequence that is
typically carried out prior to the start of nightly observing.  The
Prompt Processing system computes parameters for quality assessment of
these calibration data, and returns the QA parameters to the observing
system.  Examples of defects that could be detected are the presence
of light in a bias exposure and a deviation of a flat field from the
norm, indicating a possible problem with the flat-field screen or its
illumination.  The sequence is considered complete when processing
(which necessarily lags acquisition of the pixel data) is finished or
aborted.

\item Narrow-band flats and calibrations involving the collimated beam
projector help determine the response of the system, as a function of
frequency, over the optical surfaces.  The process of collecting these
calibrations is lengthy; the bandpass over all LSST filters (760 nm)
is large compared to the ~1nm illumination source, and operations
using the CBP must be repeated many times as the device is moved to
sample all the optical paths in the system.  The length of time needed
to collect these calibrations leads to the requirement that the Prompt
Processing system be available during the day. \\

\end{itemize}

Time for an absolutely dark dome, which is important for these
calibrations, is subject to an operational schedule.  This schedule
needs to provide for maintenance and improvement projects within the
dome.  These calibrations may be taken on cloudy nights or any other
time.  Because these operations are lengthy, and time to obtain the
calibrations quite possibly precious, prompt processing is needed to
run QA codes to help assure that the data are appropriate for
use. Note, the prompt processing system will not be used to construct
these calibrations.

Consideration of the lengthy calibrations, and the complexity of
scheduling them means that the system must be reasonably available
when needed. 


\paragraph{Operational Scenarios}

\subparagraph{Code Performance Problems} 
Should a code run longer than budgeted, and the pace of processing
fail to keep up with the pace of images, operations input is needed
because there are trade-offs, as this would affect the production of
timely QA data.  However, note that when this situation occurs no
immediate human intervention is needed.  The Prompt Processing system
provides a number of policies (which are TBD) to Observatory
Operations that can be selected via the OCS interface.  These policies
are used to prioritize the need for prompt production of QA versus
completeness of science processing, decide the conditions when science
processing should abort due to timeout, and determine how any backlog
of processing (including backlogs caused by problems with
international networking) is managed.  The policies may need to be
sensitive to details of the observation sequence.

\subparagraph{Offline Backup} 
When needed, all of this processing, including both generating QA
parameters and running science codes, can be executed under offline
conditions at the LDF at a later time.  The products of
this processing may still be of operational or scientific value even
if they are not produced in a timely manner.  Considering Alert
Production, for example, while alerts may not be transmitted in
offline conditions, transients can still be incorporated into the
portion of the L1 Database that records transients.  QA image
parameters used to gauge whether an observation meets quality
requirements can still be produced and ingested for use by the scheduler.

\subparagraph{Change Control}
Upgrades to the LSSTCam Prompt Processing Services are produced in the LSST Data Facility. Change control 
of this function is coordiniated with the Observatory, with the Observatory having an absolute say
about insertion and evaluation of changes.
