\subsubsection{Scope of Document}
This section describes the services provided to Observatory 
Operations to access data that satisfies the requirements that 
are unique to observing operations. These requirements include 
service levels appropriate for nightly operations. 

\subsubsection{Overview}

\paragraph{Description}

The Observatory Operations Data Service provides fast-access to 
recently acquired data from Observatory instruments and designated 
datasets stored in the LSST permanent archive.

\paragraph{Objective}

There is a need for regular and ad-hoc access to LSST datasets for
staff and tools working in the context of Observatory Operations. The 
quality of service (QoS) needed for these data is distinct from the general 
availability of data via the Data Backbone. Access to data provided by the 
Observatory Operations Data Service is distinguished from normal access 
to the Data Backbone in the role the data play in providing straightforward 
feedback to immediate needs that support nightly and daytime operations of 
the Observatory. Newly acquired data is also a necessary input for some of
these operations. The service must provide access methods that are compatible 
with the software access needs.

\paragraph{Operational Context}

The Observatory Operation Data Service is provided by the LSST Data
Facility to Observatory Operations, and is used by observers and
automated systems to access the data resident there. The service
provides the availability and service levels needed to support
Observatory Operations for a subset of the data that is critical for
short-term operational needs.

The Observatory Operations Data Service supplements the more general Data
Backbone by providing access to a subset of data at a QoS that is
different (and higher) than the general Data Backbone.  Less critical
data is provided to Observatory Operations by the Data Backbone, which
provides service levels provided generally to staff anywhere in the
LSST project. For general access to the data for assessment and
investigation at the Base Center, the service level is the same for any
scientist working generally in the survey.

The Observatory Operations Data Service is instantiated at the Base
Center. Therefore, the Observatory Operations Data Service does not
directly support activities which must occur when communications
between the Summit and Base are disrupted.

The service operates in-line with the Spectrograph and LSSTCam Archiving 
Services. Newly acquired raw data are first made available in the Observatory 
Operations server, and then are ingested into the Data Backbone permanent archive. 

The intent is to provide access to: 

\begin{itemize}

\item An updating window of recently acquired and produced data, and 
historical data identified by policy. An example policy is ``last week’s raw data''. 

\item Other data as specifically identified  by Observatory Operations. 
This may be file-based data or data resident in database engines within 
the Data Backbone.

\end{itemize}

A significant use case for the Observatory Operations Data Service is to
provide near-realtime access to raw data on the Commissioning Cluster.

\subparagraph{Interfaces}

File system export: The Observatory Operations Data Service provides access
via a read-only file system interface to designated computers in the
Observatory Operations-controlled enclaves. 

Butler interfaces: Use of the LSST Stack is advocated for Observatory Operations, 
and so access to this data is possible via access methods supported by the LSST stack.  
The standard access method provided by the LSST stack is through a set of
abstractions provided by a software package called the Butler. Whether the Observatory
Operations Service provides a Bulter interface is TBD.

Native interfaces: Not all needed application in the Observatory Operations context 
will use the LSST stack and will not be able to avail themselves of Butler abstractions.  
The service accommodates this need by providing files placed predictably into a 
directory hierarchy.

Http(s) interface: The Observatory Operations Data Service also exposes its 
file system via http(s). Use of the Observatory Authentication and Authorization 
system is required for this access.

\paragraph{Risks}

\begin{itemize}

\item Concern: The need includes a continuously updated window of newly created 
data, in contrast to the other Butler use cases.  How well the current set of 
abstractions work in a system that is ingesting new raw data  is unknown to the 
author.  

\item Concern: Similarly, data normally resident in databases is part of the desiderata. 
Fulfilling these desiderata include solutions from an ETL into flat files, to 
establishing mirrored databases. There are currently no actionable use cases 
for relational data. The technology to maintain subsets of relational data are distinct 
from the technologies to maintain subsets of files. It is likely that if relational data are 
needed, caches of relational data will need to be made by extract, transform and load 
into a file-format such as SQLite. 

\item Concern: This service needs to be available in TBD operational enclaves. (and limited to those enclaves).

\end{itemize}

\subsubsection{Operational Concepts}

\paragraph{Normal Operations}

As indicated the OOS is a a severs used by teh Spectrograph and LSSTCam arcihvers, and used
to continually present data to the Base Computing Endpoint. The Availablity targers for the
Observicatory operations server need to accomidat the availablity requirements of both systems.

\paragraph{Operational Scenarios}

Scheduling of changes to this system is controlled by the Observatory Change Control Authority.
