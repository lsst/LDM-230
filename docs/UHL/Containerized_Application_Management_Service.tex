\subsubsection{Scope}

Containerized Application Management Services provide an elastic capability to deploy containerized
applications. These services are provided with distinct configurations tailored for each enclave,
but are provisioned on a common pool of ITC resources residing in the Master Provisioning Enclaves at each site.

\subsubsection{Overview}

There are two instances of Containerized Application Management Services -
one at the Base Center and one at NCSA. These instances are the basis for servicing
elastic computing needs at each site and a portable abstraction for symmetric
deployment on commercial provisioning.

\paragraph{Description}

Both NCSA and the Base Center will have a containerized infrastructure that
logically partitions a pool of Kubernetes resources to various enclaves at the respective
sites, with policies that govern priorities and file systems exposed to
applications running in the context of each enclave.

At the Base Center, Containerized Application Management Services are supplied to the Commissioning
Cluster and the Chilean Data Access Center from this pool.

At NCSA, Containerized Application Management Services are supplied to the Development, Integration,
General Production, L1 and US Data Access Center enclaves from this pool.

\paragraph{Objective}

The objective of these services is to support LSST elastic operations by providing a containerized application
management system compatible with LSST requirements.

\begin{itemize}

\item Provide containerized application management to each enclave, respecting enclave-specific controls including information security and work rules.

\item Provide storage for containers and container management.

\item Provide adequate capacity for each site.

\end{itemize}

\paragraph{Operational Context}

These services use resources in the master provisioning enclave at each site and expose them
to a given enclave, implementing policies appropriate to that enclave.

\subsubsection{Operational Concepts}

\paragraph{Operational Scenarios}

An important consideration is that these resources do not have a constant
level of use within each enclave, and that over time the hardware resources
needed for elastic services in an enclave will change.

Operating conditions may change as well. For example, even with container
abstractions, it may be necessary to partition the the hardware resources  to
support two versions of an operating system.
