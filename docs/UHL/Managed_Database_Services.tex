\subsubsection{Scope}

Managed Database Services provide database access to data that reside in relational or non-relational databases
and generally meet at least one of the following criteria:
\begin{itemize}
\item Data originate outside of the LSST Data Facility, but are (or are potentially) used in Prompt and Data Release processing,
especially in the sense of data inputs needed to reproduce or refine a Prompt or Data Release computation.
\item Data originate as data produced within the Prompt  or Data Release production processes, and are meant to be retained for some period of time.
\item Data are production-related metadata.
\item Data are used as data coupling for processes involved in maintaining Prompt  and Data Release  products or other aspects of the LSST Data Facility.
\end{itemize}

\subsubsection{Overview}

In LSST, a distinction is made between patterns of storage of
data in a database engine (schema, for purposes here) and an
implementation of the schema in a database engine which stores the
data. In LSST, common schema are used and shared in many scenarios in
distinct but schema-compatible databases.

As an example, a common relational database schema can be used in
development, unit test, integration, and production, but realized in
different relational database software, e.g., SQLite3 in development
and a heavy commercial database in production.

\paragraph{Description}

\paragraph{Objective}

The primary focus of Managed Database Services is, as outlined, not the support
of developers, but the support of production and data needing custody or curation.
While some database schema design is performed in managed database services,
by no means is all schema designed by Managed Database Services. Managed Database
Services does have a role in determining the fitness for use of any schema present in
databases it operates.

\paragraph{Operational Context}

The operational context for Managed Database Services is the context
of the LSST Data Backbone within the LSST Data facility.

Part of the context is to consolidate database technologies where
appropriate.

\subsubsection{Operational Concepts}

\begin{itemize}

\item Select technology appropriate for managed database instances
\item Present a managed database service hosting the required schema
\item Support the evolution of schema in a managed database service
\item Provide the level of service needed for each managed database instance
\item Provide capacity planning
\item Provide installation
\item Provide configuration
\item Provide data migration
\item Provide performance monitoring
\item Provide security
\item Provide troubleshooting
\item Provide backup and data recovery
\item Provide data replication where needed

\end{itemize}
