\section{Scope of Document}

This document describes the operational concepts for the emerging LSST Data
Facility, which will operate the data management system as a set of services
that will be delivered by the LSST construction project. These services will be
incrementally stood up and operated by the construction project as part of
validation and verification activities within the construction project.

\section{Services for Observatory Operations}
The LSST Data Facility provides a set of services that supports specific
functions of Observatory Operations and generates Level 1 (L1) data products.
These Level 1 services include:

\begin{itemize}
\item  A Prompt Processing Service for Alert Production for wide-field and
targeted deep-drilling observing programs, including providing data support for
difference image templates and calibrations, Level 1 databases, interaction with
the alert-to-broker distribution subsystem, and providing feedback to observers.
\item  A Prompt Processing Service for assessing the quality of nightly
calibration exposures.
\item  A Prompt Processing Service for assessing exposures from the Collimated
Beam Projector, used as part of telescope optical path calibration.
\item  An “Offline” L1 Batch Processing Service, not commanded by OCS, to
facilitate catch-up processing for use cases involving extensive networking or
infrastructure outages, reprocessing of image parameters used by the Scheduler,
pre-processing data ahead of release production for broker training, and other
emergent use cases as directed by project policy.
\item  An Archiving Service for acquiring raw image data from the LSST main
camera and ingesting it into the Data Backbone (Section \ref{sect:Data Backbone Services} ).
\item  An Archiving Service for acquiring raw data from the spectrograph on the
Auxiliary Telescope and ingesting it into the Data Backbone(Section \ref{sect:Data Backbone Services} ).
\item  An Extract, Transform, and Load (ETL) Service for data stored in the
Engineering Facilities Database at the Observatory.
\item  An OCS-driven Batch Processing Service for Observatory Operations to
submit batch jobs via the OCS environment either to NCSA or to the Commissioning
Cluster at the Base Site.
\item  A QA and Base Computing Task Endpoint that allows fast and reliable access
through the QA portal to recently acquired data from the Observatory instruments
and other designated datasets, and ability to submit batch jobs, supporting
operations at the Base Center.  This endpoint is implemented as an instance of
the LSST Science Platform. See \citeds{LSE-319}.
\item  An Observatory Operations Data Service that allows fast and reliable
access to data recently acquired from LSST cameras and designated datasets held
in the Data Backbone(Section \ref{sect:Data Backbone Services} ).
\end{itemize}

The concept of operations for each of these services is described in the
following sections.

\subsection{LSSTCam Prompt Processing Services}
\subsubsection{Scope}
This section describes the prompt processing of raw data acquired from the main
LSST camera by the DM system.

\subsubsection{Overview}

\paragraph{Description}
During nightly operations, the DM system acquires images from the main LSST
camera as they are taken, and promptly processes them with codes specific to an
observing program.

\paragraph{Objective}
The LSSTCam Prompt Processing Services provide timely processing of newly
acquired raw data, including characterization of image quality of images, alert processing and delivery,
returning image parameters to the Observatory, and populating the Level 1
Database.

\paragraph{Operational Context}
Prompt Processing is a service provided by the LSST Data Facility as part of the
Level 1 system. It is presented to Observatory Operations as an OCS-commandable
device. The Prompt Processing Service retrieves crosstalk-corrected pixel data
from the main LSST camera at the Base Center, builds images, and sends them
to NCSA for prompt processing.

\subsubsection{Operational Concepts}

\paragraph{Normal Operations}

\subparagraph{Science Operations}

Science data-taking occurs on nights when conditions are suitable. For
LSST, this means all clear nights, even when the full moon brightens
the night sky. Observing is directed by an automated scheduler. The
scheduler considers observing conditions, for example, the seeing, the
phase of the moon, the atmospheric transparency, and the part of the
sky near the zenith. The scheduler is also capable of receiving
external alerts, for example, announcements from LIGO of a
gravitational wave event. The scheduler also considers required
observing cadence and depth of coverage for the LSST observing
programs.

About 90\% of observing time is reserved for the LSST “wide-fast-deep”
program. In this program, observations will be on the wide-field
two-image-per-visit cadence, in which successive observations(only for the pair)  will be
in the same filter with no slew of the telescope. However, a new
program, potentially with a new filter, a larger slew, a different
observing cadence, or a different visit structure, can be scheduled at
any moment for Special Programs. Alternative Science Visits are supposed to be per night.  

Another envisioned program is ``deep drilling'', where many more exposures
than the two exposure visit will be taken. Deep Drilling is just one example of a Special Program.  

In practice, science data-taking will proceed when conditions are
suitable. Calibration data may be taken when conditions are not
suitable for further observations, with science data-taking resuming
when conditions again become suitable. Prompt Processing can be stopped when calibration data is being taken.  

It follows that the desired behavior for science data-taking
operations is to start the Prompt Processing system at the beginning
of the night and to turn off the system after all processing for all
science observations is finished.

The operational framework for observing discloses predictions for future exposures but they cannot be relied on until the “next visit".

During science data-taking the Prompt Processing Service computes and
promptly returns QA parameters (referred to as one of the forms of “telemetry”) to the
observing system. The QA parameters are not specific to an observing
program; examples are seeing and pointing corrections derived from the
WCS. These parameters are not strictly necessary for observing to
proceed -- LSST can observe autonomously at the telescope site, if
need be. Also note that the products are quality parameters, not the
the “up-or-down” quality judgment.

The scheduler may be sensitive to a subset of these messages and may
decide on actions, but a detailed description is TBD. 
The scheduler can make use of these parameters even
if delivered quite late, since the scheduler uses historical data
in addition to recent data.

The Prompt Processing system also executes code that is specific to an
observing program. For science exposures, the code is divided into
a front-end -- that is able to compute the parameters sent back  to the
observatory with single frame processing, and a back end , Alert Production (AP), is the specific
science code that detects transient objects. 

The detected transients are passed off to another service, which
records the alert in a catalog which can be queried offline and sends
events to an ensemble of transient brokers. Data are transmitted to end
users either via feeds from an LSST-provisioned broker or via
community-provided alert brokers.

Alert Production for deep drilling and other Special Programs is TBD.  
Other observing programs
may also include AP as a science code, or may have codes of their own.

\subparagraph{Calibration Operations}

In addition to collecting data for science programs, the telescope and
camera are involved in many calibration activities. 

The Baseline Calibrations include flats and biases.  Darks are anticipated.  LSST has a Collimated Beam Projector Calibrations Device as well.  

Nominally, a
three-hour period each afternoon is scheduled for Observatory Operations
to take dark and flat calibration data. As noted above, calibration data
may be taken during the night when conditions are not suitable for
science observations. The LSST dome is specified as being light-tight,
enabling certain calibration data to be collected whenever operationally
feasible, regardless of the time of day.

Although there are standard cadences for calibration operations, the
frequency of calibration data-taking is sensitive to the stability of
the camera and telescope. Certain procedures, such as replacement of a
filter, cleaning of a mirror, and warming of the camera, may
subsequently require additional calibration operations. In general,
calibration operations will be modified over the lifetime of the
survey as understanding of the LSST telescope and camera improves.

The Prompt Processing Service computes and promptly returns quality
parameters (referred to as one of the forms of “telemetry”) to the observing system. Note
that the quality of calibrations needed for Prompt Processing science
operations may be less stringent than calibrations needed for other
processing, such as DRP.

An operations strawman, which illuminates the general need for prompt
processing, is that there are two distinct, high-level types of
calibrations.

\begin{itemize}

\item Nightly flats, biases and darks consist of a sequence of  broad-band
flatfield exposures in each camera filter, bias frames acquired from
rapid reads of an un-illuminated camera, and optional dark images
acquired from reads of an un-illuminated camera at the cadence of the
expected exposure time.  Observers will consider the collection of
these nightly calibrations 
typically carried out prior to the start of nightly observing.  The
Prompt Processing system computes parameters for quality assessment of
these calibration data, and returns the QA parameters to the observing
system.  Examples of defects that could be detected are the presence
of light in a bias exposure and a deviation of a flat field from the
norm, indicating a possible problem with the flat-field screen or its
illumination.  The sequence is considered complete when processing
(which necessarily lags acquisition of the pixel data) is finished or
aborted.

\item Narrow-band flats and calibrations involving the collimated beam
projector help determine the response of the system, as a function of
frequency, over the optical surfaces.  The process of collecting these
calibrations is lengthy; the bandpass over all LSST filters (760 nm)
is large compared to the ~1nm illumination source, and operations
using the CBP must be repeated many times as the device is moved to
sample all the optical paths in the system.  The length of time needed
to collect these calibrations leads to the requirement that the Prompt
Processing system be available during the day. \\

\end{itemize}

Time for an absolutely dark dome, which is important for these
calibrations, is subject to an operational schedule.  This schedule
needs to provide for maintenance and improvement projects within the
dome.  These calibrations may be taken on cloudy nights or any other
time.  Because these operations are lengthy, and time to obtain the
calibrations quite possibly precious, prompt processing is needed to
run quality control codes to help assure that the data are appropriate for
use. Note, the prompt processing system will not be used to construct
these master calibrations.

Consideration of the lengthy calibrations, and the complexity of
scheduling them means that the system must be reasonably available
when needed.


\paragraph{Operational Scenarios}

\subparagraph{Code Performance Problems}
The Prompt Processing system
provides a number of policies (which are TBD) to Observatory
Operations that can be selected via the OCS interface.  These policies
are used to prioritize the need for prompt production of QA versus
completeness of science processing, decide the conditions when science
processing should abort due to timeout, and determine how any backlog
of processing (including backlogs caused by problems with
international networking) is managed.  The policies may need to be
sensitive to details of the observation sequence.

\subparagraph{Offline Backup}
When needed, all of this processing, including both generating QA
parameters and running science codes, can be executed under offline
conditions at a later time.  The products of
this processing may still be of operational or scientific value even
if they are not produced in a timely manner.  Considering Alert
Production, for example, while alerts may not be transmitted in
offline conditions, transients can still be incorporated into the
portion of the L1 Database that records transients.  QA image
parameters used to gauge whether an observation meets quality
requirements can still be produced and ingested for use by the scheduler.

\subparagraph{Change Control}
Upgrades to the LSSTCam Prompt Processing Services are produced in the LSST Data Facility. Change control
of this function is coordinated with the Observatory, with the Observatory having an absolute say
about insertion and evaluation of changes.


\subsection{LSSTCam Archiving Service}
\subsubsection{Scope}

This section describes the concept of operations for archiving designated raw data acquired from the main LSST camera to the permanent archive.

\subsubsection{Overview}

\paragraph{Description}

The LSSTCam Archiving Service acquires pixel and header 
data and arranges for the data to arrive in the Observatory 
Operations Data Server and in the Data Backbone.


\paragraph{Objective}

The objective of this system is to acquire designated raw data from
the LSST main camera and header data from the OCS system, and to
place appropriately formatted data files in the Data Backbone. The
service needs to have the capability of archiving at the nominal frame
rate for the main observing cadence, and to perform ``catch up'' archiving
at twice that rate. 

\paragraph{Operational Context}

LSSTCam Archiving is a service provided by LDF as part of the
Level 1 system. It is presented to Observatory Operations as an
OCS-commandable device. The archiving system operates independently 
from related observatory services, such as data acquisition, as well as other
Level 1 services, such as prompt processing.  However, a normal
operational mode is operation of the service such that data are
ingested promptly into the permanent archive and into the Observatory Operations
Data Server for fast-access by Observatory Operations staff.


\subsubsection{Operational Concepts}

\paragraph{Normal Operations}

The LSSTCam Archiving Service runs whenever it is needed. Operational goals are
to provide prompt archiving of camera data and to provide expeditious
catch-up archiving after service interruptions.

LSSTCam data is, by default, ingested into the permanent archive
and into the Observatory Operations Data Server. However, while all
science and calibration data from the main camera require ingest into
the Observatory Operations Server, some data (e.g., one-off
calibrations, engineering data, smoke test data, etc.) may not require
archiving in the Data Backbon permanent store. Observatory Operations may 
designate data which will not be archived.

\paragraph{Operational Scenarios}

\subparagraph{Delayed Archiving}

In delayed archiving, Observatory Operations may need to prioritize
the ingestion of data into the archiving system based on operational
dependencies with the Observatory Operations Data Service. The
archiving service provides a number of policies (which are TBD) to
Observatory Operations that can be selected via the OCS interface in order to
prioritize data ingestion.

Other operational parameters of interest include rate-limiting when network 
bandwidth is a concern.

\subparagraph{Change Control}

Upgrades to the LSSTCam Archving Service are produced in the LSST Data Facility.
Change control of this function is coordiniated with the Observatory, with the 
Observatory having an absolute say about insertion and evaluation of changes. 


\subsection{Spectrograph Archiving Service}
\subsubsection{Scope}

This section describes the concept of operations for archiving raw data acquired
from instruments on the Auxiliary Telescope to the permanent archive.

\subsubsection{Overview}

\paragraph{Description}

The Auxiliary Telescope is a separate telescope at the Summit site, located on a
ridge adjacent to that of the main telescope building. This telescope supports a
spectrophotometer that measures the light from stars in very narrow bandwidths
compared to the filter pass bands on the main LSST camera. The purpose of the
spectrophotometer is to measure the absorption, which is how light from astronomical
objects is attenuated as it passes through the atmosphere. By pointing this instrument
at known “standard stars” that emit light with a known spectral distribution, it is
possible to estimate the absorption. This information is used to derive photometric
calibrations for the data taken by the main telescope.

The Auxiliary Telescope camera produces 2-dimensional CCD images, but the headers
and associated metadata are different than the LSSTCam data because spectra, not
images of the sky, are recorded. 

From the point of view of LSST Data Facility Services for Observatory Operations, the
spectrograph on the Auxiliary Telescope is an independent instrument that is controlled
independently from the main LSSTCam. Thus, the operations of and changes to LSST
Data Facility services for this instrument must be decoupled from all others.

The Auxiliary Telescope and its spectrograph are devices under the
control of the observatory control system (OCS). The spectrograph
contains a single LSST CCD. The Camera Data System (CDS) for the single CCD in the
spectrograph uses a readout system based on the LSSTCam electronics and will
present an interface for the Archiver to build files.

\paragraph{Objective}

The Spectrograph Archiving Service reads pixel data from the Spectrograph version of
the CDS and metadata available in the overall Observatory Control System and builds files.
The service archives the data in a way that the data are promptly available to Observatory
Operations via the Observatory Operations Data Service, and that the data appear in the Data
Backbone.

\subsubsection{Operational Concepts}

Archiving is under control of OCS, with the same basic operational
considerations as the CCD data from LSSTCam. Keeping
in mind the differences between the two systems, the concept of
operations for LSSTCam archiving apply (see section on LSSTCam Archiving Service).
One differing aspect could be is that these data are best organized temporally,
while some data from LSSTCam are organized spatially.

There is no prompt processing of Spectrograph data in a way that is
analogous to the prompt processing of LSSTCam data.

\paragraph{Normal Operations}

Under normal operations the Spectrograph Archiving Service is under control of the Observatory
Control System.

\paragraph{Operational Scenarios}

\subparagraph{Change Control}
Upgrades to the Spectrograph Archiving Service are produced in the LSST Data Facility.
Change control of this function is coordinated with the Observatory, with the Observatory
having an absolute say about insertion and evaluation of changes.


\subsection{EFD ETL Service}
\subsubsection{Scope}

The Engineering and Facility Database (EFD) See \citeds{LTS-210} is a system used in the
context of Observatory Operations. It contains data, apart from pixel
data acquired by the Level 1 archiving systems, of interest to LSST
originating from any instrument or any operation related to
observing. The EFD is an essential record of the activities of
Observatory Operations. It contains data for which there is no
substitute, as it records raw data from supporting instruments,
instrumental conditions, and actions taking place within the
observatory.

This section describes the concept of operations for ingesting the
EFD data into the LSST Data Backbone and transforming this data
into a format suitable for offline use.

\subsubsection{Overview}

\paragraph{Description}

The Original Format EFD, maintained by Observatory Operations, is
conceived of as a collection of files and approximately 20 autonomous
relational database instances, all using the same relational database
technology. The relational tables in the Original Format EFD have a
temporal organization. This organization supports the need within
Observatory Operations to support high data ingest and access
rates. The data in the Original Format EFD is immutable, and will not
change once entered.

The EFD also includes a large file annex that holds flat files that
are part of the operational records of the survey.

\paragraph{Objective}

The prime motivation behind the EFD Transformation Service is to be able to
relate the time series data to raw images, and to hold these quantities in
a manner that is accessible using the standard DM methods for file and
relational data access.

The baseline design calls for all of the EFD relational
and flat-file material to be ingested into what is called the Transformed
EFD.

\begin{enumerate}

\item There is a need to access a substantial subset of the Original Format
EFD data in the general context of Level 2 data production and in the
Data Access Centers. This access is supported by a query-access
optimized, separately implemented relational database, generally
called the Transformed EFD. A prime consideration is to relate the
time series data to raw images.

\item To be usable in offline context, files from the Original Format EFD
need to be ingested into the LSST Data Backbone. This ingest operation
requires provenance and metadata associated with these files.

\item Because the Original format EFD is the fundamental record related to
instrumentation, the actions of observers, and related data, the data
contained within it cannot be recomputed, and in general there is no
substitute for this data. Best practice for disaster recovery is to
not merely replicate the Original Format EFD live environment, but also
to make periodic backups and ingests to a disaster recovery system.

\end{enumerate}

\paragraph{Operational Context}

Ingest of data from the Original Format EFD into the Transformed EFD
must be controlled by Observatory Operations, based on the principle
that Observatory Operations controls access to the Original Format EFD
resources. The prime framework for controlling operations is the OCS
system. Operations in this context will be controlled from the OCS
framework.

\paragraph{Risks}

The query load applied by EFD Transformation Service  on the Original Format
EFD at the Base Center may be disruptive to the primary purpose,
serving Observatory Operations.

\subsubsection{Operational Concepts}

\paragraph{Normal Operations}

\subparagraph{Original Format EFD Operations}

Observatory Operations is responsible for Original Format EFD
operations during LSST Operations. Observatory
Operations will copy database state and related operational
information into a disaster recovery store at a frequency consistent
with a Disaster Recovery Plan approved by the LSST ISO. The LSST Data
Facility will provide the disaster recovery storage resource. The DR
design procedure should consider whether normal operations may begin
prior to a complete restore of the Original Format EFD.

If future operations of the LSST telescope beyond the lifetime of the
survey do not provide for operation and access to the Original Format
EFD, the LSST Data Facility will assume custody of the Original Format
EFD and arrange appropriate service for these data (and likely move
the center of operations to NCSA) in the period of data production
following the cessation of LSST operations at the Summit and Base
Centers.

LSST staff working on behalf of any operations department will have
access to the Original Format EFD at the Base Center for one-off
studies, including studying the merits of data being loaded into the
Transformed EFD. 

\subparagraph{EFD Large File Annex Handling and Operations}

Under control of the EFD Transformation Service, the LSST Data Facility will
ingest the designated contents of the file annex of the Original
Format EFD into the data backbone. The LSST Data Facility will arrange that these
files participate in whatever is developed for disaster recovery for the
files in the Data Backbone.

These files will also participate in the general file metadata and
file-management service associated with the Data Backbone, and thus be
available using I/O methods of the LSST stack.

\subparagraph{Transformed EFD Operations}

\begin{itemize}
\item The Transformed EFD is replicated to the US DAC and the Chilean DAC.
\item LDF will extract, transform and load into the Transformed EFD pointers to files
that have been transferred from the EFD large file annex into the Data Backbone.
\item LDF will extract, transform and load all tabular data from the
Original Format EFD into the Transformed EFDs residing in the Data Backbone at
NCSA and the Base Center. Information in Tranformed EFD that is private needs to be blocked on export.  
\item “Designated” data will include:
  \begin{itemize}
  \item Any quantities used in a production process.
  \item Any quantities designated by an authorized change control process.
  \end{itemize}
\item The information in the Transformed EFD is available to any authorized independent
DAC which may choose to host a copy.
\end{itemize}

\paragraph{Operational Scenarios}

\subparagraph{ETL Control}

The Extract, Transform and Load operation is under the control of Observing Operations.

\subparagraph{Disaster Recovery and DR Testing for the Original Format EFD}

Observing Operations will periodically test a restore in a disaster recovery scenario.

\subparagraph{Disaster Recovery and DR Testing for the Transformed EFD}

Since the Transformed EFD relational database is reproducible from the
Original Format EFD, disaster recovery is provided by a re-ingest from the
original format. DR testing includes re-establishing operations of the Transformed
EFD relational database and ETL capabilities from the Original Format EFD. Ingested
files from the file annex can be recovered by the general disaster recovery capabilities
of the Data Backbone.

\subparagraph{Change Control}

Upgrades to the EFD ETL Service are produced by the LSST Data Facility.
Change control of this function is coordinated with the Observatory, with the
Observatory having an absolute say about insertion and evaluation of changes.


\subsection{OCS-Driven Batch Service}
\subsubsection{Scope}

The OCS-driven Batch Service provides an OCS-commandable device for Observatory
Operations staff to submit batch jobs to the Commissioning Cluster, and optionally rendevous 
with a small amount of returned data via the Telemetry Gateway.

\subsubsection{Overview}

The service provides a method for OCS script to invoke Batch
production, allowing for the automation batch procesing into simple work
flows involving LSST instrumentation.


\paragraph{Description}

The servine profives OCS commands to initiate Batch Jobs and to optionally
rendevous wiht small amounts of data. 

\paragraph{Objective}

Examples of the type of workflows may be verifation of flats or wave-front sensing
application.


\paragraph{Operational Context}

The service is an OCS-commandable device which runs inder the control of
Observatory  Operations.

\subsubsection{Operational Concepts}

\paragraph{Normal Operations}

During normal operations, the OCS scripts submit batch jobs. The subject batch systems attempt to dispatch jobs at high priority. 

\paragraph{Operational Scenarios}

The promptness of the return of results varies accorsing to the ablity
of the batch system to schedule the coresponding batch job promptly.

\subparagraph{Change Control}

Upgrades to the OCS-driven Batch Service are produced by the LSST Data Facility. Change control 
of this function is coordiniated with the Observatory, with the Observatory having an absolute say
about insertion and evaluation of changes.


\subsection{Observatory Operations Data Service}
\subsubsection{Scope of Document}
This section describes the services provided to Observatory 
Operations to access data that satisfies the requirements that 
are unique to observing operations. These requirements include 
service levels appropriate for nightly operations. 

\subsubsection{Overview}

\paragraph{Description}

The Observatory Operations Data Service provides fast-access to 
recently acquired data from Observatory instruments and designated 
datasets stored in the LSST permanent archive.

\paragraph{Objective}

There is a need for regular and ad-hoc access to LSST datasets for
staff and tools working in the context of Observatory Operations. The 
quality of service (QoS) needed for these data is distinct from the general 
availability of data via the Data Backbone. Access to data provided by the 
Observatory Operations Data Service is distinguished from normal access 
to the Data Backbone in the role the data play in providing straightforward 
feedback to immediate needs that support nightly and daytime operations of 
the Observatory. Newly acquired data is also a necessary input for some of
these operations. The service must provide access methods that are compatible 
with the software access needs.

\paragraph{Operational Context}

The Observatory Operation Data Service is provided by the LSST Data
Facility to Observatory Operations, and is used by observers and
automated systems to access the data resident there. The service
provides the availability and service levels needed to support
Observatory Operations for a subset of the data that is critical for
short-term operational needs.

The Observatory Operations Data Service supplements the more general Data
Backbone by providing access to a subset of data at a QoS that is
different (and higher) than the general Data Backbone.  Less critical
data is provided to Observatory Operations by the Data Backbone, which
provides service levels provided generally to staff anywhere in the
LSST project. For general access to the data for assessment and
investigation at the Base Center, the service level is the same for any
scientist working generally in the survey.

The Observatory Operations Data Service is instantiated at the Base
Center. Therefore, the Observatory Operations Data Service does not
directly support activities which must occur when communications
between the Summit and Base are disrupted.

The service operates in-line with the Spectrograph and LSSTCam Archiving 
Services. Newly acquired raw data are first made available in the Observatory 
Operations server, and then are ingested into the Data Backbone permanent archive. 

The intent is to provide access to: 

\begin{itemize}

\item An updating window of recently acquired and produced data, and 
historical data identified by policy. An example policy is ``last week’s raw data''. 

\item Other data as specifically identified  by Observatory Operations. 
This may be file-based data or data resident in database engines within 
the Data Backbone.

\end{itemize}

A significant use case for the Observatory Operations Data Service is to
provide near-realtime access to raw data on the Commissioning Cluster.

\subparagraph{Interfaces}

File system export: The Observatory Operations Data Service provides access
via a read-only file system interface to designated computers in the
Observatory Operations-controlled enclaves. 

Butler interfaces: Use of the LSST Stack is advocated for Observatory Operations, 
and so access to this data is possible via access methods supported by the LSST stack.  
The standard access method provided by the LSST stack is through a set of
abstractions provided by a software package called the Butler. Whether the Observatory
Operations Service provides a Bulter interface is TBD.

Native interfaces: Not all needed application in the Observatory Operations context 
will use the LSST stack and will not be able to avail themselves of Butler abstractions.  
The service accommodates this need by providing files placed predictably into a 
directory hierarchy.

Http(s) interface: The Observatory Operations Data Service also exposes its 
file system via http(s). Use of the Observatory Authentication and Authorization 
system is required for this access.

\paragraph{Risks}

\begin{itemize}

\item Concern: The need includes a continuously updated window of newly created 
data, in contrast to the other Butler use cases.  How well the current set of 
abstractions work in a system that is ingesting new raw data  is unknown to the 
author.  

\item Concern: Similarly, data normally resident in databases is part of the desiderata. 
Fulfilling these desiderata include solutions from an ETL into flat files, to 
establishing mirrored databases. There are currently no actionable use cases 
for relational data. The technology to maintain subsets of relational data are distinct 
from the technologies to maintain subsets of files. It is likely that if relational data are 
needed, caches of relational data will need to be made by extract, transform and load 
into a file-format such as SQLite. 

\item Concern: This service needs to be available in TBD operational enclaves. (and limited to those enclaves).

\end{itemize}

\subsubsection{Operational Concepts}

\paragraph{Normal Operations}

As indicated the OOS is a a severs used by teh Spectrograph and LSSTCam arcihvers, and used
to continually present data to the Base Computing Endpoint. The Availablity targers for the
Observicatory operations server need to accomidat the availablity requirements of both systems.

\paragraph{Operational Scenarios}

Scheduling of changes to this system is controlled by the Observatory Change Control Authority.


\subsection{Observatory Operations QA and Base Computing Task Endpoint}
TBD
% \input{}

\section{Services for Offline Campaign Processing}

The LSST Data Facility provides specific “offline” (i.e., not coupled to
Observatory Operations) data production services to generate Level 2 data
products, as well as Level 1-specific calibration data (e.g., templates for
image differencing). Bulk batch production operations consists of executing
large or small processing campaigns that use released software configured into
pipelines to produce data products, such as calibrations and DRP products.
Processing campaigns include

\begin{itemize}
\item  Annual Release Processing: Processing of payloads of tested work flows at
NCSA and satellite sites through and including ingest of release products into
file stores, relational databases, and the Data Backbone, including system
quality assurance.
\item  Calibration Processing: processing of payload tested work flows at NCSA
and satellite sites through and including ingest of release products into file
stores, relational databases, and the Data Backbone, including initial quality
assurance. Calibration production occurs at various cadences from potentially
daily to annual, depending on the calibration data product.
\item  Special Programs and Miscellaneous Processing: processing other than
specifically enumerated.
\item  Batch framework upgrade testing: Test suites run after system upgrades
and other changes to verify operations.
\item  Payload Testing Verification and validation: of work flows from the
continuous build system on the production hardware located of NCSA and satellite
sites.
\end{itemize}

The concept of operations for batch production services serving Offline Campaign
Processing is described in the following section.

\subsection{Batch Production Services}
\subsubsection{Scope}

This section describes the operational concepts for batch production services,
which are a set of services used to provide designated offline campaign
processing.

\subsubsection{Overview}

\paragraph{Description}
Batch production service operations consists of executing large or small processing
campaigns that use released software configured into pipelines to produce data products,
such as calibrations and data release products.

\paragraph{Objective}
Batch production services execute designated processing campaigns to achieve
LSST objectives. Example campaigns include calibration production, data release
production, ``after-burner'' processing to modify or add to a data release,
at-scale integration testing, producing datasets for data investigations, and
other processing as needed. Campaign processing services provide first-order QA
of data products.

\begin{itemize}

\item A campaign is a set of pipelines, a set of inputs to run the pipelines
against, and a method of handling the outputs of the pipelines.

\item A campaign satisfies a need for data products. Campaigns produce the
designated batch data products specified in the DPDD \citedsp{LSE-163}, and
other authorized data products.

\item Campaigns can be large, such as an annual release processing, or small,
such as producing a few calibrations.

\end{itemize}

\paragraph{Operational Context}
Batch production services execute campaigns on computing resources to produce
the desired LSST data products, which are measured against first-level quality
criteria.

\subsubsection{Operational Concepts}
A pipeline is a body of code, typically maintained and originated within the
Science Operations group. Each pipeline is an ordered sequence of individual
steps. The output of one or more steps may be the input of a subsequent step
downstream in the pipeline. Pipelines may produce final end data products in
release processing, may produce calibrations or other data products used
internally within LSST operations, may produce data products for investigations
related to algorithm development, and may produce data products for testing
purposes that cannot be satisfied using development infrastructure.

A campaign is the set of all pipeline executions needed to achieve a LSST
objective.

\begin{itemize}

\item Each campaign has one or more pipelines.

\item Each pipeline possesses one or more configurations.

\item Each campaign has a coverage set, enumerating the distinct pipeline
invocations. There is a way to identify the input data needed for each
invocation.

\item Each campaign has an ordering constraint that specifies any dependencies
on the order of running pipelines in a campaign.

\item Each campaign has an adjustable campaign priority reflecting LSST priority
for that objective.

\item Each pipeline invocation may require one or more input pipeline data sets.

\item Each pipeline invocation produces one or more output pipeline data sets.
Notice that, for LSST, a given file may be in multiple data sets.

\item For each input pipeline data set there is a data handling scheme for
handling that data set in a
way that inputs are properly retrieved from the archive and made available for
pipeline use.

\item For each output pipeline data set there is a data handling scheme for
handling that data set in a way that outputs are properly archived.

\end{itemize}

The key factor in the nature of the LSST computing problem is the inherent
trivial parallelism due to the nature of the computations. This means that large
campaigns can be divided into ensembles of smaller, independent jobs, even
though some jobs may require a small number of nodes.

Batch Production Services are distinct from other services that may use batch
infrastructure, such as Development Support Services. Also, there are other
scenarios where pipelines need to be run outside the batch production service
environment.  For example, alternate environments include build-and-test,
capable desk-side development infrastructure, and ad-hoc running on central
development infrastructure.

From these considerations, the LSST Data Facility separates the concerns of a
reliable production service from these other use cases, which do not share the
concerns of production. This also allows for supporting infrastructure to evolve
independently. Example production service concerns include

\begin{itemize}

\item Supporting reliable operation of an ensemble of many campaigns, respecting
priorities.

\item Dealing with the problems associated with large-scale needs.

\item Dealing with the campaign documentation, presentation, curation and similar
aspects of formally produced data.

\end{itemize}

Computing resources are needed to carry out a campaign. Batch processing occurs
on LSST-dedicated computing platforms at NCSA and CC-IN2P3, and potentially on
other platforms. Resources other than for computation (i.e., CPU and local
storage), such as custodial storage to hold final data products and network
connectivity, are also needed to completely execute a pipeline and completely
realize the data handling scheme for input and output data sets.

Computing resources are physical items which are not always fit for use. They
have scheduled and unscheduled downtimes, and may have scheduled availability.
The management of campaigns, provided by the Master Batch Job Scheduling Service
requires:

\begin{enumerate}

\item the detection of unscheduled downtimes of resources

\item recovery of executing pipelines affected by unscheduled downtimes, and

\item best use of available resources.

\end{enumerate}

One class of potential resources  are opportunistic resources which may be very
capacious but not guarantee that jobs run to completion. These resources may be
needed in contingency circumstances. The Master Batch Job Scheduling Service is
capable of differentiating kills from other failures, so as to enable use of
these resources.

The types of computing platforms that may be available, with notes, are as
follows.

\begin{longtable}{|p{0.3\textwidth}|p{0.5\textwidth}|}\hline
\textbf{Platform Type} & \textbf{Notes} \\\hline
NCSA batch production computing system & Ethernet cluster with competent cluster
file system. \\\hline
NCSA L1 computing for prompt processing & Shared nothing machines, available
when not needed for observing operations. \\\hline
NCSA L3 computing & TBD \\\hline
CC-IN2P3 bulk computing & Institutional experience is shared nothing machines +
competent caches
and large volume storage. \\\hline
``Opportunistic'' HPC & LSST type jobs  running in allocated or in  backfill
context on HPC computers. [Backfill context implies jobs can be killed at
unanticipated times]. \\\hline
\end{longtable}

An Orchestration system is a system that supports the execution of a pipeline
instance. The basic functionality is as follows:

\begin{itemize}

\item Pre-job context:

    \begin{itemize}

    \item Supports pre-handling of any input pipeline data sets when in-job
    context for input data is not required.

    \item Pre-stages into a platform’s storage system, if available

    \item Produces condensed versions of database tables into portable
    lightweight format (e.g., MySQL to SQLite, flat table, etc.)

    \item Deals with TBD platform-specific edge services.

    \item Identities and provides for local identity on the computing platforms.

    \item Provides  credentials and end-point information for any needed LSST
    services.

    \end{itemize}

\item In-job context:

    \begin{itemize}

    \item Provides stage-in for any in-job pipeline input data sets

    \item Provides any butler configurations necessarily provided from in-job
    context.

    \item Invokes the pipeline and collects pipeline output status and other
    operational data

    \item Provides any “pilot job” functionality.

    \item Provides stage-out for pipeline output data sets when stage-out
    requires job context.

    \end{itemize}

\item Post-job context:

    \begin{itemize}

    \item Ingests any designated data into database tables.

    \item Arranges for any post-job stage out from cluster file systems

    \item Arranges for detailed ingest into custodial data systems

    \item Transmits job status to workload management, defined below.

    \end{itemize}

\end{itemize}

A Master Batch Job Scheduling Service:

\begin{itemize}

\item Considers the ensemble of available compute resources and the ensemble of
campaigns.

\item Dispatches pipeline invocations to an Orchestration System based on
resource availability and considering priority of campaigns.

\item Considers pipeline failures reported by the Orchestration System.

    \begin{itemize}

    \item Identifies errors indicative of a problem with computing resources,
    and arranges for incident report.

    \item Identifies some computational errors, and arranges for incident report.

    \item Retries failed pipeline invocations, if appropriate.

    \end{itemize}

\item Exposes progress of the campaign to relevant entities.

\item Provides appropriate logging and events (N.b. critical events can be
programmed to initiate an incident).

\end{itemize}

Quality support:

Operations are supported by the following concepts, defined as follows for this
document.

\begin{itemize}

\item Quality Assurance (QA) is what people do. This is identifying the issue
and arranging for fixes. One source of input is quality controls, described
below. Another source of input are the operational and scientific data products.

\item A Quality Control (QC) is a software artifact that produces some sort of
data that contains measure of quality. This data artifact may be

    \begin{itemize}

    \item Simply produced, recorded and not used, because it seems useful for
    some future, likely retrospective purpose.

    \item Displayed or presented for quality analysis.

    \item Fed as input into active quality control which is software that
    automatically affects the execution of a campaign.

    \item Fed into software that computes additional downstream quality control
    data.

    \end{itemize}

\end{itemize}

\paragraph{Normal Operations}

During normal operations, Batch Production Services will conduct a number of
concurrent campaigns that support LSST goals. These campaigns will be drawn from

\begin{itemize}

\item Runs to validate Data Release Processing,

\item Data Release Processing itself.

\item After-burner processing (to correct specific errors in not-yet-released
data products).

\item Calibration processing.

\item Miscellaneous processing.

\end{itemize}

While Batch Production Services will use the majority of LSST batch capability,
they may share the LSST batch infrastructure with certain Level 1 services that
require offline processing and with Level 3 batch awardees. Resource conflicts
are sorted out and expressed as priorities for each respective campaign.

The system is programmed to deal with anticipated errors. Human eye is applied
during working hours, and can be summoned when events in the underlying systems
 generate incidents.

Each campaign is monitored for technical progress, both in in the sense of
analyzing and responding to overtly flagged errors, and general monitoring and
human assessment of the overall performance of the service.

First-order Quality Assurance is as follows:

\begin{enumerate}

\item Quality controls are considered by an LSST Data Facility Production
Scientist and other staff. Data Facility staff apply any standard authorized
mitigations, such as reprocessing, flagging anomalies, etc. The Production
Scientist within the LSST Data Facility understands the full suite of quality
controls, alerts Science Operations group to anomalies, and collaborates in
diagnosis and mitigation of problems, as requested.

\item The service provided by the LSST Data Facility Production Scientist uses
operational and scientific acumen to assess the data products at a first level,
in addition to monitoring the extant quality controls. Particular attention is
paid to

    \begin{enumerate}

    \item operationally critical data (e.g., next night’s flats needed for L1
    processing)

    \item a processing campaign that is resource intensive, hence expensive to
    redo (or has expensive consequences)

    \item known problematic output data sets that are not adequately covered by
    existing quality controls.

    \item known problematic input data sets not adequately covered by existing
    quality controls.

    \end{enumerate}

\end{enumerate}

Close collaboration is maintained between first order quality assurance and the
broader scientific quality assurance in the project. Information obtained from
first order quality assurance is continuously fed back to Science Operations.

Campaign closeout provides that all outputs are in final form, documentation and
other artifacts have been produced, and all parties are actively notified about
the status of a campaign.

\paragraph{User Genearted Products Batch Services}
Authorized Science Users will also use batch services.   This is a future development area.
\paragraph{Development Support Batch Services}
Small-scale ad-hoc batch services are also needed for use by developers and scientists on the operations team.
Automated batch services used for regularly-scheduled test processing (i.e. continuous integration)is also needed.
This is future development area.



\paragraph{Operational Scenarios}

\subparagraph{Initiate campaign}

Campaigns are initiated in response to an LSST objective, by specifying an
initial set of pipelines, a coverage set, and an initial priority. The Batch
Production Service is consulted with a reasonable lead time. Consistent with
LSST processes, pipelines can be modified or added (for example, in the case of
after-burners) during a campaign. These changes and additions are admitted when
the criteria of change control processes are satisfied, including

\begin{itemize}

\item relevant build-and test criteria

\item the impact of resource-intensive campaigns is approved and understood

\item production-scale test campaigns

\end{itemize}

\subparagraph{Terminate failed campaign}
Reasons for a campaign failure will be documented and submitted to Science
Operations for review. Deletion of data products needs to be scheduled so that
it occurs after the review is completed. This includes backing out files, materials
from databases, and other production artifacts from the Data Backbone, and
maintaining production records as these activities occur.

\subparagraph{Pause campaign}
Stop a long running campaign from proceeding allowing for TBD interventions.

\subparagraph{Deal with problematic campaign}
LSST is a large system. Pipelines will evolve and be maintained. There will be
the campaigns, described in the operations documents. It is the nature of the
system that as issues emerge extra resources will be needed to provide focused
scrutiny on aspects of production for some pipeline. In many cases problems will
be resolved by bug fixes, or addressed by quality controls and changes to
processes. Any system needs to support mustering focused effort on quality
analysis that is urgent, and lacks an adequate basis for robust quality controls.
The LSST Data Facility Batch Production Services staff contribute effort to to solve
these problems, in collaboration with Science Operations (or other parties
responsible for codes).

\subparagraph{Deal with defective data}
Production data may be deemed defective immediately as the associated pipelines
terminate or after a period of time when inspection processes run. Such data need
to be marked such that they will not be included in release data and will be set aside
for further analysis.

\subparagraph{Deal with sudden lack (or surplus) in resources}
As noted above, for large scale computing, the amount of resource available to
support all campaigns will vary due to scheduled and unscheduled outages.

The technical system responds to an increase or decrease in resources by running
more or few jobs, once the workload manager is aware of the new level of
resources. The technical system responds to hardware failures on a running job
in just like any other system -- with the ultimate recovery being to  delete an
partial data and retry, while respecting the priorities of the respective campaigns.


\section{Data Access Hosting Services for Authorized Users}
The LSST Data Facility provides authorized users and sites access to data via a
set of services that are integrated with the overall Authentication and
Authorization (AA) System. These services are hosted by LSST Data Facility at
the US and Chilean Data Access Centers and will include hosting elements of the
LSST Science Platform.

\subsection{User Data Access Services}
Service hosting elements of the LSST Science Platform.
\subsubsection{Scope}
This section describes the concept of operations for hosting Data Access Center services in the US and Chile.

\subsubsection{Overview}

\paragraph{Description}

Hosting of user data access services consists of:
\begin{itemize}

\item Providing Kubernetes for the LSST Science Platform, including the SUIT portal, Jupyter Notebooks, and DAX.

\item Providing a Qserv instance for each data release.

\item Loading a Qserv with the released data to the extent not already done during data release production.

\item Providing support for users entering data into their MyDBs and entering sharded data into Qserv to the extent that is supported, as well as providing backup and disaster recovery for data entered into these databases.

\item Providing user file space and backups and disaster recovery for the file space.

\item Providing read-only access to data release products identified in the DPDD and other designated data resources, such as the reformatted EFD.

\item Providing batch computing capabilities.

\item Providing a second-tier help desk support to the primary help desk.

\item Providing for special administration considerations of the Chilean DAC outlined in the Chilean MoU.

\item Providing a security enclave appropriate for protecting LSST resources from actions of the user community.

\item Integrating the LSST Authentication and Authorization system.

\item Providing service-level monitoring to measure and assure a designated level of service.

\item Providing users with a default level of access to computing resources.

\item Providing users with a computing award with the designated amount of computing resources.

\end{itemize}

\paragraph{Objective}

The objective is to host the data access services at the required availability for authorized users.

\paragraph{Operational Context}

The operational context is the LSST LDF infrastructure at the Base and at NCSA. An important element of the context is the elasticity of the supporting computing infrastructure. While, for example, the US Data Access Center consist of about 10\% of the computing capacity at NCSA, it is desirable that the constructed system deliver to the end users 10\% of the computing capacity averaged over a year, borrowing resources from the production system when demand on the Data Access Center is high, and returning resources to the production system during times when demand is low.

It does not seem required that the US and Chilean Data Access Centers be mirrors of each other, as this implies capacities may need to be nearly equal, which is not a requirement of the Chilean MoU.



\subsection{Bulk Data Distribution Service}
Service providing bulk data download to sites supporting groups of users.
\subsubsection{Scope}


Provide large-scale access to organizations.

\subsubsection{Overview}

\paragraph{Description}

Provide access to organized collaborations of authorized users and
participating institutions supporting authorized users to bulk data
specified in the DPDD, organized as files.

Offered at NCSA only.

\paragraph{Objective}

\begin{enumerate}

\item Provide large-scale data access to organizations entrusted to hold large
amounts of data during its proprietary period.

\item Provide large-scale data access to designated organizations for data after
proprietry data rights have expired.

\end{enumerate}

\paragraph{Operational Context}

\paragraph{Risks}

\begin{enumerate}

\item An organization may attempt to access data other than the data agreed to.

\item If the form of delvered files are not technology neutral, users of the  service may not find the
service useful.

\item Information may be lost, compared to information available through the Data Access Centers. 

\end{enumerate}

\subsubsection{Operational Concepts}

\paragraph{Normal Operations}


\paragraph{Operational Scenarios}

\begin{enumerate}

\item Forwarding of Raw data to CC-IN2P3 is logically part of the data backbone functioality.

\item The return of data products computed at satellite computing facilties is parlt of the workflow system.

\end{enumerate}

\subparagraph{Change Control}

Offine change control applies.



\subsection{Hosting of Feeds to Brokers}
The LSST Data Facility hosts the alert distribution system and supports users of
the LSST mini-broker, as well as providers of community brokers.
\subsubsection{Scope}
This section describes the concept of operations for hosting the system that distributes transient events to the LSST-provided and community-provided brokers..

\subsubsection{Overview}

\paragraph{Description}

Alert Distribution consists of:
\begin{itemize}

\item Receiving alerts from instances of the Alert Production Pipeline or backlogs of alerts from offline L1 production.

\item If needed, formatting received alerts into broker-specific syntax and forwarding the alerts to services which propagate alerts to the community. This includes  community brokers selected by LSST and also  the LSST alert “mini broker”, which is a limited functionality broker the LSST construction project is providing.  

\item Operating instances of the LSST alert “mini broker,” which includes accepting filters from individual authorized users.

\item Providing status information about these activities to the community.

\item Providing a clear interface for trouble-shooting, monitoring, and other operational matters.

\item Providing an audit trail sufficient for troubleshooting, monitoring and statistics.

\item Separating the concerns of alert generation from alert distribution.

\item Is capable of providing service to the prompt processing verion and various batch configurations of Alert Processing.

\end{itemize}

\paragraph{Objective}

The objective is to provide a configurable layer that receives events from instances of the alert production pipeline and delivers alerts to event brokers ultimately resulting in end-user consumption while supporting the various operations scenarios enumerated below. This layer decouples event generators from the complexity of policy-defined event distribution. 

\paragraph{Operational Context}
Alerts originate in the LSST Alert Production pipeline. The alert distribution system is configured into the overall pipeline process control system when running production modes at NCSA.

The context for this document is the distribution of the alerts between the following operational entities. 

\begin{itemize}
\item A running alert pipeline which outputs alerts to an interface configured to pass alerts to the alert distribution service.

\item The authoring interface for each supported instance ofthe alert broker. Alert Brokers transmit alerts to subscribed users, according to their own Service Level Agreements with their users.  There are community-provided alert brokers and an LSST-provided “mini broker”.  

\item There may be feeds to multiple instantiations of a given broker. A use case that illustrates this need is to support testing upgrades of brokers, and the planned multiple instantiations of the LSST mini-broker.

\item As the design evolves, possibly serving as an intermediate buffer between the AP codes (which cannot block in a lengthy manner) and the database ingest for the L1 database which stores records of the alert.

\item Records and presents broker instances with the ability to ingest presented alerts; and to record the number of drops, and other operational matters.  

\item A validation end point, which “looks like” a broker, but records the alerts sent to it, as a component for smoke testing, and other testing and operational needs of the alert distribution service itself.

\end{itemize}

\paragraph{Risks}
While the dominant method foreseen for alert distribution is the International Virtual Observatory Alliance (IVOA) VOEvent mechanism, practical brokers need to mature significantly to handle the LSST data rate. Moreover, it is likely that specialized brokers will serve specific astronomical interests as brokers can apply further science classification. Each event packet is large; not all information is of interest to every science topic. Providing for a way to reduce the packet size emitted at NCSA or allow brokers to filter packets before they are emitted at NCSA are risk mitigating features that need to be considered, and supported if consistent with budget. Some thought should be given in design for providing alerts to non-IVOA compliant entities. Attention should be paid to protocol and other issues related to scaling.

\subsubsection{Operational Concepts}

\paragraph{Normal Operations}
The normal operating scenario is Prompt Processing. In this scenario the alert distribution needs to be part of an overall system which normally presents an alert to a broker within 60 seconds of the data being prepared. 

The distribution system needs

\begin{itemize}

\item Only to present an alert to a broker instance, one and only once.

\item For brokers that can ingest in a timely way, introduce no more than a well-stated delay between production of the alert and presentation to the broker.  

\item Consider the effects of a given broker that is unable to ingest the stream of events subscribed to in a timely way.

\item Need to protect the throughput of any one feed due to broker misbehavior from misbehavior of other brokers.

\item Needs to accept alerts from the alert production pipelines.

\end{itemize}

\paragraph{Operational Scenarios}

Smoke testing:  Smoke tests are end-to-end tests of the L1 system. These tests are

\begin{itemize}

\item available to Observatory Operations to verify an L1 service is functioning.

\item used to validate changes to an L1 service.

\end{itemize}

Testing of alert distribution is an element of smoke testing. The validation endpoint is used for this test. Testing of feeds to brokers is desired, but not required, as a valid system does not depend on the functioning of external components.

Offline processing:  Offline alert distribution refers to distribution of alerts outside of the context prompt alert processing.

\begin{itemize}

\item When online processing fails, alert distribution may be configured into the system if offline processing occurs sufficiently recently after data is taken.

\item Will likely occur when Alert Production algorithms change, due to the need to develop training sets for brokers with algorithms that need training, and where the software change in Alert Production may have affected that training. In this case alerts produced by the new software need to be conveyed appropriately to the brokers. 

\item Testing of upgrades of the alert distribution service itself with downstream brokers. 

\end{itemize}

Availability: The availability requirements for the Alert Production system are quite high. The availability of Alert Distribution is a component of that availability.   

\begin{itemize}

\item Alert distribution needs to be tested as a separate component from Alert Production AP software.

\item Needs to be instantiated as part of the L1 Complete Test Stand. 

\end{itemize}

Broker test/broker support: LSST has a notion of a limited number of supported brokers. In this model, the set of authorized brokers will change over time. Each broker will have a service level (or similar) agreement with the LSST project that provides information about the needed level of interface. The alert distribution system needs to provide a vocabulary of support actions, in addition to providing the real time stream of alerts. This support is envisioned as:

\begin{enumerate}

\item Alert replay, including full-rate replay, to support resolution of end-to-end problems involving rate.

\item Concurrent operations of two feeds to support major upgrades of a broker’s infrastructure. 

\item Pushing training sets to community-provided alert brokers, for example in the case that our data model changes, and the classification algorithms in the target broker need to see training set data, processed by the new LSST algorithms.

\end{enumerate}



\section{Data, Compute, and IT Security Services}
The LSST Data Facility provides a set of general IT services which support the
LSST use-case-specific services mentioned in previous sections. These
``undifferentiated heaving lifting'' services include

\begin{itemize}
\item Data Backbone Services providing file ingestion, management, movement
between storage tiers, and distribution to sites.
\item Managed Database Services providing database administration for all
database technologies and schema managed for the project.
\item Batch Computing and Data Staging Environment Services providing batch
capabilities on each LSST-provided platform at NCSA and the Base Center.
\item Containerized Application Management Services providing elastic
capabilities for deploying containerized applications at NCSA and the Base Center.
\item Network-based IT Security Services providing project-wide intrusion
detection, vulnerability scanning, log collection and analysis, and incident and
event detection, and verification of controls.
\item Authentication and Authorization (AA) Services providing central management of
identities, supporting workflows and various authentication mechanisms, and
operating AA endpoints at the Summit, Base, and Archive Sites.
\end{itemize}

The concept of operations for each of these services is described in the
following sections.

\subsection{Data Backbone Services}
\subsubsection{Scope}

The Data Backbone is a set of data storage, management and movement
resources distributed between the Base Center and NCSA. The
scope of the Data Backbone includes both files and data maintained by relational
and other database engines holding the record of the survey and used by
L1, L2, and Data Access Center services.

The Data Backbone provides read-only data service to the US and Chilean
DACs, but does not host data stores where DAC users create state. This is
done to create a hard and easily enforceable separation of technologies,
where no flaw in a DAC can corrupt the data produced by L1 and L2
production systems. For example, DAC resources such as Qserv and user databases,
colloquially know as MyDBs, are provisioned in the context of a data
access center, not the Data Backbone.

The Data Backbone ensures that designated data sets are replicated at both sites.
The data backbone provides an enclave environment that is oriented toward
protecting data by management, operational, and technical controls,
including processes such as maintaining disaster recovery copies.

\subsubsection{Overview}

\paragraph{Description}

Files in the data backbone are presented as file system mounts and data access
services. Database-resident data are presented as managed database services.

\paragraph{Objective}

\begin{itemize}

\item Replication of designated file data within LSST Data Facility sites
at NCSA and the Base Center.

\item Replication of designated relational tables and data maintained
in other database engines at NCSA and the Base Center.

\item Implementation of policy-based flows to the disaster recovery stores. At
the time of this writing, disaster recovery stores include the NCSA tape archive, CC-IN2P3,
and commercial providers.

\item Ingest of images produced by the Spectrograph, ComCam, and LSSTCam instruments.

\item Ingest of the Engineering and Facility Database and associated Large File Annex.

\item Ingest of data products from L1 and L2 production processing.

\item Ingest of data from TBD other sources, approved by a change control process.

\item Serving data to L1, L2, and other approved production processes.

\item Serving data to the US and Chilean Data Access Centers.

\item Integrity checking and other curation activates related to data integrity and
continuity of operations.

\end{itemize}

\paragraph{Operational Context}

\subsubsection{Operational Concepts}

\subparagraph{Files}

Files in the Data Backbone possess path names which are subject to change
through the lifetime of the LSST project, which at the time of this writing is
seen as serving the last data release through 2034.

Robust identification of a file involves
\begin{itemize}
\item Obtaining a logical file name through querying metadata and provenance.
\item Possibly migrating a file from a medium where the file is not directly accessible,
such as tape, to a medium where the file is accessible.
\item Selecting a distinct instance of the file from possibly many replicas.
\item Accessing the file though an access method such as a file system mount or Http(s).
\end{itemize}

The project has identified several caches of data that are used in production circumstances.
The distinguishing circumstances for these caches involve quality of service requirements for
performance and availability. Absent sophisticated QoS in file systems, performance requirements
are met by controlling access to the underlying storage via caching. Availability is assured by
decoupling the cache from the database providing metadata, provenance, and location information.
Application-level cache management provides path names within the cache to the application.

Casual use of data for short periods may rely on knowledge such as file paths, but is subject
to disruption when paths are re-arranged, or should the underlying storage technology change,
such as introduction of object stores.


\subparagraph{Data Managed by Databases}

Replication of database information is specific to the database technology involved.
Databases identified as holding permanent records of the survey are in the Data Backbone
in the sense that they are instantiated in the context of a security enclave with management,
operational, and technical controls needed to assure preservation of this data, and that the
principal concern of enclave management is that data reside at the Base and at NCSA
driven by business need.

\subparagraph{Managed Lifetime of Data}
Data in teh data backdone are preseres and replicated under processes
guided by an overall Data Management Policy.  Elements of the policy
include specifications for replication suportign disasater recovery,
and the permissiblity of deleing fiel based data, and purgingin
database records.  Examples of deletion of data likley in such a
policy are: Removal of production data from faulty production campaigns,
removal of data produced in the production system that are not part
of the retained record of the LSST suvey, and removal of data in
shakedown and debug processing.

\paragraph{Operational Scenarios}

\subparagraph{Availablity and Change Management}

Catalog-based access systems such as indicated for the Data Backbone are
limited by database availability as well as the availability of the file
store and its access methods.

Time-critical applications involving the Observatory Operations Data
Service and access to L1 templates for prompt processing protect
themselves by having caches as described above.


\subsection{Managed Database Services}
\subsubsection{Scope}

Managed Database Services provide database access to data that reside in relational or non-relational databases
and generally meet at least one of the following criteria:
\begin{itemize}
\item Data originate outside of the LSST Data Facility, but are (or are potentially) used in Prompt and Data Release processing,
especially in the sense of data inputs needed to reproduce or refine a Prompt or Data Release computation.
\item Data originate as data produced within the Prompt  or Data Release production processes, and are meant to be retained for some period of time.
\item Data are production-related metadata.
\item Data are used as data coupling for processes involved in maintaining Prompt  and Data Release  products or other aspects of the LSST Data Facility.
\end{itemize}

\subsubsection{Overview}

In LSST, a distinction is made between patterns of storage of
data in a database engine (schema, for purposes here) and an
implementation of the schema in a database engine which stores the
data. In LSST, common schema are used and shared in many scenarios in
distinct but schema-compatible databases.

As an example, a common relational database schema can be used in
development, unit test, integration, and production, but realized in
different relational database software, e.g., SQLite3 in development
and a heavy commercial database in production.

\paragraph{Description}

\paragraph{Objective}

The primary focus of Managed Database Services is, as outlined, not the support
of developers, but the support of production and data needing custody or curation.
While some database schema design is performed in managed database services,
by no means is all schema designed by Managed Database Services. Managed Database
Services does have a role in determining the fitness for use of any schema present in
databases it operates.

\paragraph{Operational Context}

The operational context for Managed Database Services is the context
of the LSST Data Backbone within the LSST Data facility.

Part of the context is to consolidate database technologies where
appropriate.

\subsubsection{Operational Concepts}

\begin{itemize}

\item Select technology appropriate for managed database instances
\item Present a managed database service hosting the required schema
\item Support the evolution of schema in a managed database service
\item Provide the level of service needed for each managed database instance
\item Provide capacity planning
\item Provide installation
\item Provide configuration
\item Provide data migration
\item Provide performance monitoring
\item Provide security
\item Provide troubleshooting
\item Provide backup and data recovery
\item Provide data replication where needed

\end{itemize}


\subsection{Batch Computing and Data Staging Environment Services}
\subsubsection{Scope}

Batch Computing and Data Staging Services provide primitives
used by the Master Batch Job Scheduling Service. Batch Computing
and Data Staging Services are provides in a distinct implementation
for that is tailored for each batch system deployment.

\subsubsection{Overview}

Batch Computing and Data Staging Services are provided at NCSA and the Base Center.

Analogous (but not identical) services are provided by MoU to the LSST
Data Facility by CC-IN2P3, as well as by any commercial batch
provisioning and agency resources, such as XSEDE.

\paragraph{Description}

Both NCSA and the Base Center will have a core batch infrastructure that
uses batch system logic to partition a pool of batch resources to
various enclaves at the respective sites, with policies that govern
priorities and file systems exposed for batch nodes running in the
context of each enclave.

At the Base Center, Batch Services are supplied to the Commissioning
Cluster and the Chilean Data Access Center from this pool.

At NCSA, Batch Services are supplied to the Development, Integration,
General Production, L1 and US Data Access Centers enclaves from this pool.

An additional pool of batch resources at each site is drawn from idle nodes
in the core Kubernetes provisioning. An enhanced goal is the unification of
resource management of the Kubernetes nodes and the batch pool.

Data staging refers to mechanisms needed to move data, primarily files, between the
Data Backbone and the storage used by batch programs. This may be as simple
as a copy operation between mounted file systems, or as complex as a staging
via http or FTP.

\paragraph{Objective}

Batch Computing and Data Staging Services support LSST batch
operations by providing a batch system supported by data movement primitives.

\begin{itemize}

\item Provide a batch scheduler.

\item Provide any enclave-specific resources. An example is distinct head
nodes for different enclaves.

\item Provide enclave-specific configurations, including configurations
needed for information security and work processes.

\item Integrate ITC into the batch system.

\end{itemize}

\paragraph{Operational Context}

Batch Computing and Data Staging Services use resources in the master provisioning
enclave and expose them to a given enclave, implementing policies appropriate to that enclave.

\subsubsection{Operational Concepts}

\paragraph{Operational Scenarios}

An important consideration is that these resources do not have a constant
level of use within each enclave, and that over time the hardware resources
needed for batch operations in an enclave will change.

Operating conditions may change as well. For example even with container
abstractions, it may be necessary to partition the batch resources to
support two versions of an operating system.

Somewhat analogously, the batch system may opportunistically use idle nodes provisioned
for elastic Kubernetes computing.

Lastly, NCSA has substantial resources for prompt processing, such as alert
production. Scheduling jitter and performance may preclude using a single batch
scheduler for general offline production and prompt processing. Batch Computing
and Data Staging Services covers having multiple scheduler instances.


\subsection{Containerized Application Management Services}
\subsubsection{Scope}

Containerized Application Management Services provide an elastic capability to deploy containerized
applications. These services are provided with distinct configurations tailored for each enclave,
but are provisioned on a common pool of ITC resources residing in the Master Provisioning Enclaves at each site.

\subsubsection{Overview}

There are two instances of Containerized Application Management Services -
one at the Base Center and one at NCSA. These instances are the basis for servicing
elastic computing needs at each site and a portable abstraction for symmetric
deployment on commercial provisioning.

\paragraph{Description}

Both NCSA and the Base Center will have a containerized infrastructure that
logically partitions a pool of Kubernetes resources to various enclaves at the respective
sites, with policies that govern priorities and file systems exposed to
applications running in the context of each enclave.

At the Base Center, Containerized Application Management Services are supplied to the Commissioning
Cluster and the Chilean Data Access Center from this pool.

At NCSA, Containerized Application Management Services are supplied to the Development, Integration,
General Production, L1 and US Data Access Center enclaves from this pool.

\paragraph{Objective}

The objective of these services is to support LSST elastic operations by providing a containerized application
management system compatible with LSST requirements.

\begin{itemize}

\item Provide containerized application management to each enclave, respecting enclave-specific controls including information security and work rules.

\item Provide storage for containers and container management.

\item Provide adequate capacity for each site.

\end{itemize}

\paragraph{Operational Context}

These services use resources in the master provisioning enclave at each site and expose them
to a given enclave, implementing policies appropriate to that enclave.

\subsubsection{Operational Concepts}

\paragraph{Operational Scenarios}

An important consideration is that these resources do not have a constant
level of use within each enclave, and that over time the hardware resources
needed for elastic services in an enclave will change.

Operating conditions may change as well. For example, even with container
abstractions, it may be necessary to partition the the hardware resources  to
support two versions of an operating system.


\subsection{Network-based IT Security Services}
\subsubsection{Scope}

This section describes network-based operational information security services
supporting the Observatory Operations and the LSST Data Facility.

\subsubsection{Overview}

\paragraph{Description}

The LSST Network-based IT Security Service provides technical controls for
operational security assurance. These controls provide data that support the
LSST Master Information Security Plan and IT security processes such as incident
detection and resolution.

\paragraph{Objective}

The objective of the Network-based IT Security Service is to provide:

\begin{itemize}
\item Network Security Monitoring, including monitoring of high-rate data
connections for data transfer across the LDF system boundaries (but excluding
certain high rate transfers, such as the Prompt service access to the Camera
Data System), including deployment of technologies for Active Response and
Blocking of Attacks.
\item Vulnerability Management for computers and application software in the
enclaves.
\item The technical framework to facilitate efficient Incident Detection and
Response, including central log collection/event correlation for security purposes.
\item Management of certain access controls, such as firewalls and bastion hosts,
used for administrative access.
\item Host-based intrusion detection clients deployed on end systems as
appropriate.
\item Security configuration management and auditing to baseline standards.
\end{itemize}

\paragraph{Operational Context}

The general approach to operational information security is that there is a LSST
Information Security Officer (ISO) who reports to the Head of the LSST
construction project, and will transition to report to the Head of the Operations
project.

The ISO drafts a Master Security Program \citedsp{LPM-121} plan, which the Head
approves of as appropriately mitigating the Information Security risk. The Head
then assumes responsibility for the residual risk of the plan. This is the
security risk that remains, given faithful execution of the plan. The ISO
oversees implementation and evolution of the plan, seeing that it is faithfully
implemented and noting when mitigation and changes are needed. The ISO does any
required staff work for the Head; for example, running staff training. The ISO
is informed of and keeps records on security incidents, and is responsible for
evolution of the security plan and evolution of security threats.

The ISO is responsible for a Information Security Response Team, which deals
with actual or latent potential breaches in information security. The Incident
Response Team is made of a set of draftees from the various operations
departments, with the draft weighted towards departments with expertise and
responsibility for critical operations and critical information security needs.

 The ISO runs the annual security plan assessment. The management of each
 construction subsystem and operations department is responsible for annual
 revision of a Departmental Security Plan that complies with the Master Plan.
 These departmental plans include

\begin{itemize}
\item A comprehensive list of IT assets, applications and services.
\item A list of security controls the department applies to each asset
(technical and operational).
\item A list of controls supplied by others that are relied on.
\end{itemize}

These controls apply to all offered services and all supported ITC. Reporting
is easiest if the systems offered are under good configuration control. Under a
good system, the security plans are living and updated by an effective change
control process.

Verification: The ISO oversees a group that provides network-based security
services described in the Objective part of this concept of operations.

A general approach to LSST-specific networking is the use of software-defined
networking. This provides for isolation of networking supporting security enclaves.
In particular, this allows for the separation of critical infrastructure for
Observatory Operations and the LSST Data Facility from office or other routine
networking.

The context for these security enclaves cover the following production services
in the LSST project, though other enclaves may join if feasible and desired by
the relevant operational partners. These networks may participate in this
infrastructure, but are currently seen as the responsibilities of AURA and NCSA.

\begin{longtable}{|p{0.3\textwidth}|p{0.5\textwidth}|} \hline
\textbf{Production Service} & \textbf{Security Enclave} \\\hline
Prompt Services & Split Between NCSA and Chile  \\\hline
Batch Production services & NCSA portion, excluding satellite center \\\hline
US Data Access Center Services & NCSA \\\hline
Critical Observing Enclave Services & Summit and Base Center \\\hline
Chilean Data Access Center Services & Base Site \\\hline
Data Backbone Services & Base Center and NCSA \\\hline
\end{longtable}

\paragraph{Risks}

These are the standard elements of an information security infrastructure which
are needed for a credible IT security project. Certain elements of the system
are near the state of the art due to the data rates involved. Lack of credible
infrastructure in this area will be seen as a flaw in the overall construction
plan, preparing the LSST MREFC for operations.

\subsubsection{Operational Concepts}

\paragraph{Normal Operations}

The following elements provide the functionality needed to implement the
network-based security elements of the LSST Master Information Security Plan:

\begin{itemize}

\item Intrusion Detection Systems (IDS) detect patterns of network activity that
indicate attacks on systems, compromise of systems, violations of Acceptable Use
of systems, abuse of systems, and other security-related matters.

\item Vulnerability Scanning detects software services with vulnerable
configurations or unpatched versions of software via network fingerprinting. The
system scans designated systems subject to a black-list. In addition to scanning
for vulnerabilities, port scanning for firewall audits and ARP scanning for
network asset management can also be conducted.

\item Central Log Collection and Event Generation collects syslogs and other
designated logs for storage (making logs invulnerable for an attacker to modify)
and processes the logs to detect signatures indicating a compromise or poor
security practices.

\item Firewalls and bastion hosts provide a layer of active security. A typical
use of a bastion host is to provide a layer of security between networks used to
administer computers and more general networks.

\item Host-based Intrusion Detection complements network monitoring by detecting
actions within a system not visible from the network with tools such as auditd
and OSSEC. This component also monitors the filesystems and checks for filesystem
integrity.

\item Active Response blocks communication with entities outside the observation site
networks. This component is typically used to block “bad actor” entities outside
the observation site network.

\item Central Configuration Management enforces a baseline security and configuration
on all systems.

\end{itemize}

New systems being deployed must be ``hardened'' to a security baseline and
vetted by security professionals before moving into operations or after major
configuration changes.

\paragraph{Operational Scenarios}

Vulnerability scanning periodically assesses designated ports on designated
computers sensing vulnerabilities. An example of when this service is applied in
a crucial case is assessing the effectiveness of the program of work patching a
critical vulnerability.

Intrusion detection can detect, for example, an attempt to compromise a system.
The detection system interacts with the active response system to cut off the
attacker’s access to the computer. The intrusion detection system can also be
used to aid in the investigation of an attack during the incident response and
handling of a security incident.

Host-based Intrusion Detection checks for attacks against a host from the
perspective of the host. Examples include multiple failed remote logins as reported
by the host, or reports of file system changes that do not accompany an approved
request for change or do not fall within a maintenance window.

The networks at the Observatory site must be monitored by intrusion detection
systems. Acceptable IDS solutions include Bro and Snort. These systems must be
able to handle the traffic load from various network segments at 10GB to 100GB
speeds. The IDS systems must be placed at strategic locations and account for
any expansion or changes in the network without the need to completely retool
the IDS systems.

The information produced by the system is accessible by LSST staff involved in
LSST information security, “landlords” hosting systems, and other parties with a
valid interest in the data, to the extent required by the site’s specific security plans.

Active Response is typically referred to as a Black Hole Router (BHR) since it peers
with the border routers on a network and offers the shortest router to
destinations being blocked.  Quagga and ExaBGP are two examples of BHR software.

The central Configuration Management System will enforce a security baseline and
configuration on all systems. Examples of this technology are Puppet and Chef.
In the event that Windows systems will be deployed onsite, a system enforcing
Group Policy Objects and WSUS for patching will have to be available. It is
required that this system would also be the Domain Controller using Active
Directory and federate with LSST’s Unified Identity Management system.

The central log collectors are responsible for collecting and archiving all logs
collected as described in the previous section. The collectors must be able to
store at minimum six months of logs with a rotating windows deleting the oldest
logs to maintain disk space. In addition to the log collectors, there is a SEIM/analysis
system. This system is used for real-time log alerts, searches, and
visualization. ElasticSearch, Kibana, and OSSEC are three examples of such
software. This server spools a copy of the logs from the central collectors but
may not be able to keep the full time window due to overhead or log metadata
storage.

All systems, both workstations and servers, are required to send system logs to
a central collector. For Linux systems, syslog must be configured to send a copy
of all logs in realtime to the central collector. For Windows systems, software
such as Snare will be installed to send Windows Event logs to the central
collector using the syslog protocol. Other alternatives exist for log collection
such as logstash an open source log collection tool that can be used to collect
logs from a wide variety of platforms.

Network devices are also required to send system logs to the central collector.
Note that this is different than any network logs a switch or router would send.
The system logs refer to events such as device logins, configuration changes, and
other system specific events. Note that this is a requirement only if the device
has this feature available.

Network devices such as routers or firewalls that are placed on the ingress/egress
points of a vlan or the network must send firewall or router ACL logs to the
central collector.

It is best practice for network devices to also send netflow to a central
collector. However, if netflow is collected and forwarded to the central
collector it must not be at the expense of network or device performance.

Any other devices not classified as a server, workstation, or networking device
must be configured to send logs to the central collector if this feature is
available. An example of a device that falls into this category is a VOIP appliance
 or a VPN appliance.


\subsection{Authentication and Authorizations Services}
See \citeds{LSE-279}.

\section{ITC Provisioning and Management}

ITC is managed in distinct enclaves. Enclaves are defined based on administrative
and security controls, and operational availability requirements. Enclaves may
span geographic sites, with elements in both the Base Facility in La Serena and
at NCSA. Enclaves may share computing and other resources. 
Central administration is operated by NCSA staff, including remote administration of the Base Facility, 
with support staff in Chile.

See \citeds{LDM-129}, the LSST Data Facility Logical Infrastructure Technology and Communications (ITC) Design Document, for more information.

\section{Service Management and Monitoring}

The LSST Data Facility provides a set of services supporting overall management of services, as well as monitoring infrastructure which collects information about running services for service delivery, incident response, planning for future upgrades, and supporting change control. Service management processes are drawn from the ITIL IT Service Management vocabulary.

\subsection{Service Management Processes}
\subsubsection{Overview}
This section briefly describes functions and processes of service
management that are implemented across all service and ITC layers of
the LSST Data Facility. These elements were drawn from the Information
Technology Infrastructure Library (ITIL) which is an industry-standard
vocabulary for IT service management.

IT Service Management processes include

\begin{enumerate}

\item Service Design: Building a service catalog and arranging for changes to the service offering, including internal supporting services.

\item Service Transition: Specifying needed changes, assessing the quality of proposed changes,
and controlling the order and timing of inserting changes into the system.

  \begin{itemize}
 
  \item \emph{Change Management} provides authorization for streams of changes to be requested, for the insertion of changes into the reliable production system, and for the assessment of the success of these changes.

  \item \emph{Release Management} interacts with project producing a specific change to ensure that
a complete change is presented to change management for approval into the live system. Examples areas that are typically a concern are accompanying documentation and security aspects.

  \item \emph{Configuration Management} provides an accurate model of the components in the live system sufficient to understand changes, and support operations.
 
  \end{itemize}

\item Service Delivery: operating the current set and configuration of production services. Service delivery processes must satisfy the detailed service delivery concepts presented elsewhere in this document.

  \begin{itemize}
  
  \item \emph{Incident Response} is invoked when a service does not
  performed as specified accordign to its version.  The goal of
  incident respose is to allow to work to continue, either by
  restoring the service to its original working form, or by providing
  a work-around.  Each LSST service has a level of incidnet 
  response support, which depend on the operational criticality.  For
  example, support of observatory services is more critical that support
  of a developer test stand.

  \item \emph{Request Response} Request reponse covers both structured
  requests which are well supported by a workflow, and unstructured
  requests.  An example of structured requests are to on-board or
  off-board a user. Examples of  unstructured requests are to answer a
  question about the implementation of a service, or to produce an
  informative report.
 
  \item \emph{Problem Management} Not all issues with a service can be
  addressed at a given time. These issues are aggrigated into a problem
  backlog. The Problem back log contains unaddressed, valid issues
  know to service users, and problems generated by internal staff.
  Problem managemenet is the process where items are selected from the
  backlog to be addressed, root cause determined, followed by a change
  request to implement a fix.
  
  \end{itemize}
  
\end{enumerate}


\subsection{Service Monitoring}
\subsubsection{Scope}
This section describes operational concepts of systems-level and service-level
monitoring for services operated by the LSST Data Facility.

\subsubsection{Overview}

\paragraph{Description}

The service monitoring system is the source of truth for the health and status
of all operational services within its scope. The monitoring system deals with
quality controls related to service delivery. These data have both retrospective
and real-time uses:

\begin{itemize}

\item Acquires data from subordinate monitoring systems within components that
are not bespoken LSST software. These monitoring systems may have an API, log
files, SNMP, and TBD other interfaces.

\item Acquires data from native LSST interfaces, including interfacing to the
logging package (lsst.log),  Prompt logging, Prompt events,
scoreboards (Redis), TBD Qserv, and data from other independent packages.

\item Probes services from monitoring agents and ingests quality control
parameters.

\item Synthesizes new quality control data from existing quality control data
(for example, correlating a series of events before generating an event that
will issue a page).

\item Can generate events based on performance or malfunction which can trigger
incident response for services and ITC, including to a non-NCSA incident
response software.

\item Can generate reports used for problem management, availability management,
capacity management, vendor management and similar processes.

\item Provides dashboard (or comfort) displays satisfying the use cases defined
below.

\item Provides for instantiation of displays anywhere within the LSST
operational environment (concerns porting vs. remote display, paint display with
high latency).

\item Provides for publicly visible displays and displays visible only to those
authorized by the LSST Authentication and Authorization system.

\item Is sensitive to dynamic deployment of services to ITC resources.

\item Is sensitive to modes of deployment and test, integration, development
when generating alerts, painting displays, and recording data for retrospective
use (concerns segregation and separation).

\item Is itself highly reliable and available.

\item Provides for disconnected operations between geographic sites (Summit,
Base and NCSA) and enclaves (e.g., Observing Critical and non-Observing
Critical).

\end{itemize}

\paragraph{Objective}

The set of services and infrastructure relied on by LSST Data Facility
operations is inherently distributed due to the distributed deployment of the
LDF services. Reliable operations of LDF services involves components
instantiated (at least) in Chile, at NCSA, and at CC-IN2P3, as well as the
networks between these sites.

A dataset based on the operational characteristics of the facilities, hardware,
software and other elements of service infrastructure is needed to support
service management, service delivery, service transition, and ITC-level
activities, as well as to provide health and status information to the users of
the systems. This dataset must be substantially unified, so that all activities
are supported by a single source of truth. From a unified dataset, for example,
staff concerned with availability management of a service can obtain records
that consistently reflect availability information generated by incident response
activities, while staff concerned with capacity management can obtain
information on how capacity is provided by ITC activities.

In general, service management needs both a subset of the data that is needed
for ITC management and data which may not be supplied by traditional ITC
monitoring. Examples of data not supplied by ITC monitoring include the
end-to-end availability of a service that tolerates hardware faults, user-facing
comfort displays which address specific areas of interest, and controls that
monitor data flow into disaster recovery stores for consistency with creation of data.

\paragraph{Operational Context}

LDF services rely on ITC hosted at NCSA, the Chilean Base Center, satellite
computing centers, test stands at LSST Headquarters, wide area networks, and
possibly other sources of infrastructure. Each of these sources possesses
organization-specific (non-uniform) ITC monitoring and service management
information on which LDF services rely. In all cases the LSST Data Facility needs
to centrally acquire sufficient data to provide for management of LDF services,
while minimizing coupling to the ITC or service provisioning from these sites.
The coupling should be defined in an internal Service Level Agreement (SLA) or
similar written instrument.

\subsubsection{Operational Concepts}

\paragraph{Normal Operations}

\subparagraph{Example: Prompt Products Services}

Prompt Products services are instantiated at NCSA and the Chilean Base Center. The
services the LDF relies on that may provide monitoring information are described
in the table below. The monitoring system may acquire additional essential data
by agents, consistent with SLAs and systems engineering best practice.

\begin{longtable}{|p{0.5\textwidth}|p{0.5\textwidth}|}
\caption{Sources of Monitoring Data} \\\hline
\textbf{Reliance} & \textbf{Subordinate monitoring interfaces provided} \\\hline
WAN from Chilean border router to UIUC & Wide area network activity area of LDF \\\hline
Network transit from Chilean L1 infrastructure subsystem to Chilean border router & Observatory Operations \\\hline
Network transport on UIUC campus to L1 installation area in NPCF & U of I networking, NCSA networking \\\hline
OCS interfaces (bridge, telemetry injection) & Observatory Operations \\\hline
CDS interface & Observatory Operations \\\hline
Base Center computing room resources & Observatory Operations, Computing Facility manager \\\hline
Local assistance in Chile & TBD (if any) \\\hline
ITC for Prompt system, exclusive of reliances listed otherwise & LDF ITC group or relied upon NCSA groups \\\hline
NCSA/NPCF facility resource management & NCSA/NPCF facility management \\\hline
Service-specific code and service-level performance as a part of the overall system, component-level aspect of L1 internals & Interfaces provided by LDF software group \\\hline
\end{longtable}

\begin{longtable}{|p{0.3\textwidth}|p{0.3\textwidth}|p{0.4\textwidth}|}
\caption{Uses of Monitoring Service Data Products} \\\hline
\textbf{Entity} & \textbf{Need} & \textbf{Notes} \\\hline
Incident response & Events indicating service faults. & TBD these directly generate notifications (page), (and have the right filtering semantics) \\\hline
Problem management & Incident information and information about marginal or near-miss events detected. & \\\hline
Observing Operations, DPP Staff, and LSST HQ & Comfort displays indicating real-time status of services used. & NCSA staff should be able to see the same information as Observatory Operations staff to prevent confusion in incident response. It is important to note that monitoring relied on by Observatory Operations in Chile needs to be independent of NCSA. Monitoring at each site needs to operate and provide appropriate subsets of information to each site, should the connectivity between sites be disrupted. \\\hline
Alert users & Information about when alerts are being exported and flows to various broker-like entities. & \\\hline
Availability management & Queries, reports and displays focused on historical contributions to failures by reliance. & \\\hline
Capacity management & Queries, reports and displays focused on historical usage of resources. & \\\hline
Contract and SLA management & Queries, reports, and displays of quantities related to performance, e.g., response times, quality of materials or services. & \\\hline
ITC staff & Supplemental information to ITC monitoring. & \\\hline
\end{longtable}

\subparagraph{Example: Wide Area Networks}

Many of the hardware components which make up the WAN will be managed by
different entities (ISPs) based on who owns the particular section of the
network. Typically all ISPs run their own SNMP network to monitor the health of
the devices.

\paragraph{Operational Scenarios}

A predefined hierarchy of roles which will include different levels of users as
listed below:

\begin{itemize}

\item Generic User
\item System Admininistrator
\item Science User
\item LSST DM Admininistrator
\item Hardware Operator (ISPs)
\item Camera Control System Administrator
\item Observatory Control System Administrator
\item Super User

\end{itemize}

The level of access and response capabilities will be as defined in the
user-profile. In the case of a ``Generic User,'' it may be necessary only to show
if the LSST system is up and running. There can be a graphical representation of
the status of the systems and subsystems.

For a ``Super User'' who will have access to detailed status information on the
systems and subsystems, will be able to see in-depth event history and status
reports (through log-scraping and fault databases). The Super User will also be
able to access the logs database through the same portal.

In-between levels of access will be defined as per the definition of the roles
and responsibilities of the user.

